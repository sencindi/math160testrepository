\documentclass{ximera}

\input{../preamble.tex}

\title{Evaluating Limits with Indeterminate Forms}

\begin{document}

\begin{abstract}
%Stuff can go here later if we want!
\end{abstract}
\maketitle

\begin{sectionOutcomes}
After completing this section, you should be able to do the following.

\begin{itemize}
\item Describe what is meant by the form of a limit.
\item Identify limits of determinate forms and indeterminate forms of the $\frac{0}{0}$ type.
\item Distinguish between determinate and indeterminate forms.
\item Evaluate limits of the form $\frac{0}{0}$.
\end{itemize}
\end{sectionOutcomes}

\phantom{text}%%% Making vertical spaces


%---Prereq notes/links----------%
\textbf{Skills you may want to brush up on first} \\ To be ready to achieve these objectives, you may need to review the following trigonometry and algebra topics: 
\begin{itemize}
    \item \href{https://ximera.osu.edu/math160fa17/m160prerequisites/PreRequisiteXards/U6MultiplyingAndFactoringPolynomials/6.1MultiplyingPolynomials/6.1TitlePage/6.1TitlePage}{Multiplying Polynomials}
    \item \href{https://ximera.osu.edu/math160fa17/m160prerequisites/PreRequisiteXards/U6MultiplyingAndFactoringPolynomials/6.2FactoringPolynomials/titlePage}{Factoring}
    \item \href{https://ximera.osu.edu/math160fa17/m160prerequisites/PreRequisiteXards/U6MultiplyingAndFactoringPolynomials/6.3SimplifyingRationalFunctions/titlePage}{Simplifying rational functions}
    \item \href{https://ximera.osu.edu/math160fa17/m160prerequisites/PreRequisiteXards/U6MultiplyingAndFactoringPolynomials/6.3SimplifyingRationalFunctions/titlePage}{Trigonometric Functions} (Especially writing them in terms of $\sin(x)$ and $\cos(x)$ and applying trigonometric identities)
\end{itemize}


\end{document}
