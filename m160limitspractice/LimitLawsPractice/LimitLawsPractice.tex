\documentclass[handout]{ximera}
%\documentclass[10pt,handout,twocolumn,twoside,wordchoicegiven]{xercises}
%\documentclass[10pt,handout,twocolumn,twoside,wordchoicegiven]{xourse}

%\author{Steven Gubkin}
%\license{Creative Commons 3.0 By-NC}
\input{../../preamble.tex}

\outcome{Practice Limits.}
   

\title{Limit laws practice}

\begin{document}
\begin{abstract}
Here is an opportunity for you to practice limit laws and the Squeeze Theorem.
\end{abstract}
\maketitle

\begin{exercise}
  Can this limit be directly computed by limit laws?
  \[
  \displaystyle\lim_{x\to -3}\frac{x^2+5x+6}{x+2} 
  \]
  \begin{multipleChoice}
    \choice[correct]{yes}
    \choice{no}
  \end{multipleChoice}
  \begin{question}
    Compute:
    \[
    \displaystyle\lim_{x\to -3}\frac{x^2+5x+6}{x+2} \begin{prompt} =\answer{0}\end{prompt}
    \]
    \begin{feedback}
      Since $f(x)=\frac{x^2+5x+6}{x+2}$ is a rational function and
      $\displaystyle\lim_{x\to -3} x+2 =-1 \neq 0$, you can use the quotient law.  The limit of the numerator and denominator can then be calculated using the limit laws because the numerator and denominator are both polynomials. 
    \end{feedback}
  \end{question}
\end{exercise}

\begin{exercise}
  Can this limit be directly computed by limit laws?
  \[
  \displaystyle\lim_{x\to -3}\frac{x^2+5x+6}{x+3} 
  \]
  \begin{multipleChoice}
    \choice{yes}
    \choice[correct]{no}
    
    \begin{feedback}[correct]
      Since $f(x)=\frac{x^2+5x+6}{x+3}$ is a rational function, you may want to use the quotient law; however, $\displaystyle\lim_{x\to -3} x+3 = 0$, so you cannot use this limit law!  Because the quotient law cannot be used, this limit cannot be evaluated with the limit laws. 
    \end{feedback}
    
  \end{multipleChoice}

\end{exercise}

\begin{exercise}
  Can this limit be directly computed by limit laws?
  \[
  \displaystyle\lim_{x\to -1} \frac{1}{|x|}
  \]
  \begin{multipleChoice}
    \choice[correct]{yes}
    \choice{no}
  \end{multipleChoice}
  \begin{question}
    Compute:
    \[
    \displaystyle\lim_{x\to -1} \frac{1}{|x|} \begin{prompt} =\answer{1}\end{prompt}
    \]
    \begin{feedback}
      Since $f(x)=\frac{1}{|x|}$ is a rational function, we would like to use the quotient law.  Before doing so, we must check that the limit of the denominator is not equal to $0$.  When $x < 0, |x| = -x$, so $\displaystyle\lim_{x\to -1} |x| = \displaystyle\lim_{x\to -1} -x = -(-1) \neq 0$ and so the quotient law can be utilized. The limit of the the numerator can also be calculated because the numerator is a constant function. 
    \end{feedback}
  \end{question}
\end{exercise}

\begin{exercise}
  Can this limit be directly computed by limit laws?
  \[
  \displaystyle\lim_{x\to 0} \frac{1}{|x|}
  \]
  \begin{multipleChoice}
    \choice{yes}
    \choice[correct]{no}
    
     \begin{feedback}
      Again, because $f(x)=\frac{1}{|x|}$ is a rational function, we would like to use the quotient law.  However, $\displaystyle\lim_{x\to 0} |x| = 0$, so the limit of the denominator is $0$, and the quotient law cannot be used.  We need techniques beyond the limit laws to evaluate this limit. 
    \end{feedback}
    
  \end{multipleChoice}
\end{exercise}

\begin{exercise}

Fill in the blanks to evaluate the following limit using the limit laws. 
  \[
  \displaystyle\lim_{x\to 4} \frac{(3x-7)(\sqrt{x})}{5x}  \]
First, you could use the quotient law to conclude that 

\[
  \displaystyle\lim_{x\to 4} \frac{(3x-7)(\sqrt{x})}{5x} = \frac{\displaystyle\lim_{x\to 4} (3x-7)(\sqrt{x})}{\displaystyle\lim_{x\to 4} 5x} \]
by the \wordChoice{\choice{Constant Multiple Law}\choice{Sum/Difference Law}\choice{Product Law}\choice[correct]{Quotient Law}}, which you can use because 

\[
  \displaystyle\lim_{x\to 4} 5x \neq \answer{0}. \]
  
The numerator can be now be re-written using the \wordChoice{\choice{Constant Multiple Law}\choice{Sum/Difference Law}\choice[correct]{Product Law}\choice{Quotient Law}}: 

\[ \frac{\displaystyle\lim_{x\to 4} (3x-7)(\sqrt{x})}{\displaystyle\lim_{x\to 4} 5x} = \frac{\left(\displaystyle\lim_{x\to 4} 3x-7 \right) \left(\displaystyle\lim_{x\to 4}\sqrt{x}\right)}{\displaystyle\lim_{x\to 4} 5x} \]

Now, you can evaluate all of the limits that appear individually: 

\begin{itemize}

\item $\displaystyle\lim_{x\to 4} 3x-7 = \answer{5}$ by the \wordChoice{\choice[correct]{Constant Multiple Law}\choice[correct]{Sum/Difference Law}\choice{Product Law}\choice{Quotient Law}} and the \wordChoice{\choice[correct]{Constant Multiple Law}\choice[correct]{Sum/Difference Law}\choice{Product Law}\choice{Quotient Law}}.

\item $\displaystyle\lim_{x\to 4} \sqrt{x} = \answer{2}$

\item $\displaystyle\lim_{x \to 4} 5x = \answer{20}$

\end{itemize}

Combining all this information together, you can conclude that 

\[
  \displaystyle\lim_{x\to 4} \frac{(3x-7)(\sqrt{x})}{5x} = \answer{\frac{1}{2}}.  \]

\end{exercise}

\begin{exercise}

Suppose that $\displaystyle\lim_{x \to -4} Q(x) = 3, \displaystyle\lim_{x\to -4} R(x) = 7,$ and $\displaystyle\lim_{x\to -4} S(x) = -2$.  Using this information, evaluate the following limits. 

\begin{itemize}

\item $\displaystyle\lim_{x \to -4} \left( Q(x) -5S(x) - R(x) \right) = \answer{6}$

\item $\displaystyle\lim_{x \to -4} Q(x)\left(S^2(x) + 5 \right) = \answer{27}$

\item $\displaystyle\lim_{x \to -4} \frac{2S(x) + Q(x)}{R(x)} = \answer{\frac{-1}{7}}$

\end{itemize}

\begin{hint}

Note: $S^2(x) = (S(x))^2$

\end{hint}

\end{exercise}

\begin{exercise}

Use the fact that $\displaystyle\lim_{x \to 0} \frac{\sin x}{x} = 1$ to evaluate the following limits. 

\begin{itemize}

\item $\displaystyle\lim_{x \to 0} \frac{ \sin x}{5x} = \answer{\frac{1}{5}}$

\item $\displaystyle\lim_{x \to 0} \frac{\tan x}{x} = \answer{1}$

\end{itemize}
\end{exercise}

\begin{exercise}
Can this limit be directly computed by limit laws?
  \[
  \displaystyle\lim_{x\to 0} \left(-x^6\cos\left(\frac{\pi}{x}\right)\right)
  \]
  
  \begin{multipleChoice}
      \choice{Yes}
      \choice[correct]{No}
  \end{multipleChoice}
  
\begin{exercise}
  
Right - the limit laws will not help you in this situation.  Here is a graph of this function to help you see what's going on.  Use the + button to zoom in.
  \[
   \graph{f(x)=-x^6*\cos\left(\left(\frac{\pi}{x}\right)\right)}
  \]
  
Based on the graph, you would predict that 

\[
  \displaystyle\lim_{x\to 0} \left(-x^6\cos\left(\frac{\pi}{x}\right)\right) = \answer{0}.
  \]
\end{exercise}
\begin{exercise}

Let's verify this prediction using the Squeeze Theorem.  Fill in the blanks to complete the problem. 
 
First, you could notice that 

\[ \answer{-1} \leq \cos\left(\frac{\pi}{x}\right) \leq \answer{1} \]

so that
   
\[ \answer{-x^6} \leq -x^6\cos\left(\frac{\pi}{x}\right) \leq \answer{x^6}. \]

Before moving on, graph the two functions you entered in the last two answer boxes using the Desmos graph above.  Do these two graphs ``squeeze" $-x^6\cos\left(\frac{\pi}{x}\right)$ at $x=0$?

Now, notice that 

\[ \displaystyle\lim_{x \to \answer{0}} \answer{-x^6} = \answer{0} = \displaystyle\lim_{x \to \answer{0}} \answer{x^6}\], 

so the Squeeze Theorem allows us to conclude that 

\[ \displaystyle\lim_{x \to 0}  -x^6\cos\left(\frac{\pi}{x}\right) = \answer{0} \]

\end{exercise}
\end{exercise}





\end{document}
