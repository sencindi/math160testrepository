\documentclass[handout]{ximera}
%\documentclass[10pt,handout,twocolumn,twoside,wordchoicegiven]{xercises}
%\documentclass[10pt,handout,twocolumn,twoside,wordchoicegiven]{xourse}

\input{../../preamble.tex}

\outcome{Practice Limits.}
   

\title{Limits with infinity practice}

\begin{document}
\begin{abstract}
Here is an opportunity for you to practice evaluating limits that involve infinity.  
\end{abstract}
\maketitle

\begin{exercise}
Your goal in this exercise is to evaluate $\lim_{x \to -3} \frac{1}{(x+3)^3}$.  First, consider the corresponding one-sided limits. 

$$\lim_{x \to -3^-} \frac{1}{(x+3)^3} \text{ and } \lim_{x \to -3^+} \frac{1}{(x+3)^3}$$

As $x$ approaches $-3$ from the left:  

\begin{itemize}

\item The function $1$ approaches $\answer{1}$, which is a \wordChoice{\choice{positive}\choice[correct]{negative}} number. 

\item $(x+3)^3$ approaches a \wordChoice{\choice{positive}\choice[correct]{negative}} number. 

\end{itemize}

Therefore, 

 \[ \lim_{x \to -3^-} \frac{1}{(x+3)^3} = \answer{-\infty}.\]
 
As $x$ approaches $-3$ from the right: 

\begin{itemize}

\item The function $1$ approaches $\answer{1}$, which is a \wordChoice{\choice{positive}\choice[correct]{negative}} number. 

\item $(x+3)^3$ approaches a \wordChoice{\choice{positive}\choice[correct]{negative}} number. 

\end{itemize}

Therefore, 

 \[ \lim_{x \to -3^+} \frac{1}{(x+3)^3} = \answer{-\infty}. \]
 
Putting all of this information together, you can conclude that

\[ \lim_{x \to -3} \frac{1}{(x+3)^3} = \answer{-\infty}. \]

\end{exercise}


\end{document}
