\documentclass[handout]{ximera}
%\documentclass[10pt,handout,twocolumn,twoside,wordchoicegiven]{xercises}
%\documentclass[10pt,handout,twocolumn,twoside,wordchoicegiven]{xourse}

\input{../../preamble.tex}

\outcome{Practice Limits.}
   

\title{Limits with infinity practice}

\begin{document}
\begin{abstract}
Here is an opportunity for you to practice evaluating limits that involve infinity.  
\end{abstract}
\maketitle

\begin{exercise}
Your goal in this exercise is to evaluate $\displaystyle\lim_{x \to -3} \frac{1}{(x+3)^3}$ (if it exists).  First, consider the corresponding one-sided limits:

$$\lim_{x \to -3^-} \frac{1}{(x+3)^3} \text{ and } \lim_{x \to -3^+} \frac{1}{(x+3)^3}.$$

As $x$ approaches $-3$ from the left:  

\begin{itemize}

\item The function $1$ approaches $\answer{1}$, which is a \wordChoice{\choice[correct]{positive}\choice{negative}} number. 

\item $(x+3)^3$ approaches a small \wordChoice{\choice{positive}\choice[correct]{negative}} number. 

\end{itemize}

Therefore, 

 \[ \lim_{x \to -3^-} \frac{1}{(x+3)^3} = \answer{-\infty}.\]
 
As $x$ approaches $-3$ from the right: 

\begin{itemize}

\item The function $1$ approaches $\answer{1}$, which is a \wordChoice{\choice[correct]{positive}\choice{negative}} number. 

\item $(x+3)^3$ approaches a small \wordChoice{\choice[correct]{positive}\choice{negative}} number. 

\end{itemize}

Therefore, 

 \[ \lim_{x \to -3^+} \frac{1}{(x+3)^3} = \answer{\infty}. \]

Putting all of this information together, you can conclude that

\[ \lim_{x \to -3} \frac{1}{(x+3)^3} = \answer{DNE}. \]

\begin{exercise}

From the above limit calculations, you can say that $\frac{1}{(x+3)^3}$ \wordChoice{\choice[correct]{does}\choice{does not}} have a \wordChoice{\choice{horizontal}\choice[correct]{vertical}} asymptote at $x =-3$. 

\end{exercise}

\end{exercise}

\begin{exercise}
Your goal in this exercise is to evaluate $\displaystyle\lim_{x \to 7} \frac{-3x+2}{(x-7)^6}$ (if it exists).  First, consider the corresponding one-sided limits:

$$\displaystyle\lim_{x \to 7^-} \frac{-3x+2}{(x-7)^6} \text{ and } \displaystyle\lim_{x \to 7^+} \frac{-3x+2}{(x-7)^6}.$$

As $x$ approaches $7$ from the left:  

\begin{itemize}

\item The function $-3x+2$ approaches $\answer{-19}$, which is a \wordChoice{\choice{positive}\choice[correct]{negative}} number. 

\item $(x-7)^6$ approaches a small \wordChoice{\choice[correct]{positive}\choice{negative}} number. 

\end{itemize}

Therefore, 

 \[ \displaystyle\lim_{x \to 7^-} \frac{-3x+2}{(x-7)^6} = \answer{-\infty}.\]
 
As $x$ approaches $7$ from the right: 

\begin{itemize}

\item The function $-3x+2$ approaches $\answer{-19}$, which is a \wordChoice{\choice{positive}\choice[correct]{negative}} number. 

\item $(x-7)^6$ approaches a small \wordChoice{\choice[correct]{positive}\choice{negative}} number. 

\end{itemize}

Therefore, 

 \[ \displaystyle\lim_{x \to 7^+} \frac{-3x+2}{(x-7)^6} = \answer{-\infty}. \]

Putting all of this information together, you can conclude that

\[ displaystyle\lim_{x \to 7} \frac{-3x+2}{(x-7)^6} = \answer{-\infty}. \]

\begin{exercise}

From the above limit calculations, you can say that $\frac{-3x+2}{(x-7)^3}$ \wordChoice{\choice[correct]{does}\choice{does not}} have a \wordChoice{\choice{horizontal}\choice[correct]{vertical}} asymptote at $x =7$. 

\end{exercise}

\end{exercise}




\end{document}
