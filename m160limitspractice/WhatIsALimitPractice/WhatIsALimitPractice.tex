\documentclass[handout]{ximera}
%\documentclass[10pt,handout,twocolumn,twoside,wordchoicegiven]{xercises}
%\documentclass[10pt,handout,twocolumn,twoside,wordchoicegiven]{xourse}

%\author{Steven Gubkin}
%\license{Creative Commons 3.0 By-NC}
\input{../../preamble.tex}

\outcome{Practice Limits.}
   

\title{Limits practice}

\begin{document}
\begin{abstract}
Here's an opportunity for you to practice limits.
\end{abstract}
\maketitle

\begin{exercise}
  Evaluate the expressions by referencing the graph of $f(x)$ below. Write DNE if the limit or function value does not exist.
  
\begin{center} \includegraphics[scale=0.5]{limgraph.png} \end{center}

\begin{enumerate}
\item $\displaystyle\lim_{x\to -3^-} f(x) =\answer{2}$  

\item $\displaystyle\lim_{x\to -3^-} f(x) = \answer{2}$ 

\item $\displaystyle\lim_{x\to -3} f(x) = \answer{2}$ 

\item $f(-3) = \answer{2}$

\item $\displaystyle\lim_{x\to -1^-} f(x) =\answer{0}$ 

\item $\displaystyle\lim_{x\to -1^-} f(x) = \answer{0}$ 

\item $\displaystyle\lim_{x\to -1} f(x) = \answer{0}$ 

\item $f(-1) = \answer{DNE}$

\item $\displaystyle\lim_{x\to 0^-} f(x) =\answer{1}$ 

\item $\displaystyle\lim_{x\to 0^-} f(x) = \answer{3}$ 

\item $\displaystyle\lim_{x\to 0} f(x) = \answer{DNE}$ 

\item $f(0) = \answer{3}$

\item $\displaystyle\lim_{x\to 2^-} f(x) =\answer{3}$ 

\item $\displaystyle\lim_{x\to 2^-} f(x) = \answer{-2}$ 

\item $\displaystyle\lim_{x\to 2} f(x) = \answer{DNE}$ 

\item $f(2) = \answer{DNE}$

\item $\displaystyle\lim_{x\to 4} f(x) = \answer{0}$ 

\end{enumerate}

\end{exercise}

\begin{exercise}

Consider the statement below, and then indicate whether it is sometimes, always, or never true.

``If $f(10)$ exists, then $\displaystyle\lim_{x\to 10} f(x) = f(10)$"

This statement is \wordChoice{\choice[correct]{sometimes}\choice{always}\choice{never}} true.

\begin{hint}

Use the data you gathered in the previous exercise to help you answer this question!  

\end{hint}

\end{exercise}

\begin{exercise}

Consider the statement below, and then indicate whether it is sometimes, always, or never true.

``If $\displaystyle\lim_{x\to 10^-} f(x) = 4$ and $\displaystyle\lim_{x\to 10} f(x)$ exists, then $\displaystyle\lim_{x\to 10^-} f(x) = 4.$"

This statement is \wordChoice{\choice{sometimes}\choice[correct]{always}\choice{never}} true.

\begin{hint}

Use the data you gathered in the previous exercise to help you answer this question!  

\end{hint}

\end{exercise}

\begin{exercise}

Consider the statement below, and then indicate whether it is sometimes, always, or never true.

``The function value at a point and the value of the limit at the same point must always be equal."

This statement is \wordChoice{\choice[correct]{sometimes}\choice{always}\choice{never}} true.

\begin{hint}

Use the data you gathered in the previous exercise to help you answer this question!  

\end{hint}

\end{exercise}

\begin{exercise}
Consider $f(x) = x^2$. Fill in the 
  tables below:
  \[
  \begin{array}{l|l}
    x      & f(x)      \\ \hline
    1.1    & \begin{prompt}\answer{1.21}\end{prompt}\\
    1.01   & \begin{prompt}\answer{1.0201}\end{prompt}\\
    1.001  & \begin{prompt}\answer{1.002001}\end{prompt}\\
    1.0001 & \begin{prompt}\answer{1.00020001}\end{prompt}\\
    1.00001 & \begin{prompt}\answer{1.00020001}\end{prompt}\\
    1.000001 & \begin{prompt}\answer{1.0000200001}\end{prompt}
  \end{array}
  \]
  Based on the data you have gathered, we might predict that 
  \[
  \lim_{x\to 1} x^2 = \answer{1}
  \]
  
    \begin{feedback}
    As discussed previously, neither tables nor graphs can ever tell us for certain what a limit is. However, sometimes they can help ``guess'' the limit. In this case the graph of $f(x)$ is somewhat more helpful:
\[
\graph{sin(1/x)}
\]
  Zoom into the origin. What is happening with the function values? 
  
  We see that $f(x)$ oscillates ``wildly'' as $x$ approaches $0$, and hence does not approach any one number.
  \end{feedback}

\end{exercise}

\end{document}
