\documentclass[handout]{ximera}
%\documentclass[10pt,handout,twocolumn,twoside,wordchoicegiven]{xercises}
%\documentclass[10pt,handout,twocolumn,twoside,wordchoicegiven]{xourse}

%\author{Steven Gubkin}
%\license{Creative Commons 3.0 By-NC}
\input{../../preamble.tex}

\outcome{Practice Limits.}
    

\title{Continuity exercises}

\begin{document}
\begin{abstract}
Here is an opportunity for you to practice using the definition of continuity. 
\end{abstract}
\maketitle

\begin{exercise}

Consider the function $f(x) = \frac{x+4}{x^2-16}.$  Is this function continuous on $\RR = (\infty, \infty)?$  

\begin{multipleChoice}
    \choice{Yes}
    \choice[correct]{No}
\end{multipleChoice}

\begin{exercise}
Now that you know this function is not continuous everywhere, select all of the $x$ values at which this function is not continuous. 

\begin{selectAll}
    \choice[correct]{$x = 4$}
    \choice{$x=-16$}
    \choice{$x=16$}
    \choice[correct]{$x=-4$}
    \choice{$x=0$}
\end{selectAll}

\begin{exercise}

You got it - $f$ is discontinuous at $x=-4$ and $x=-4$.  Let's use limits and function values to determine what type of discontinuity $f$ has at $x=-4$.

First, make a prediction.  Any prediction you make is correct because it's what you think currently, so take a guess!  At $x=-4$, I predict that $f$ has a(n) \wordChoice{\choice[correct]{removable discontinuity/hole}\choice[correct]{jump discontinuity}\choice[correct]{infinite discontinuity}\choice[correct]{oscillating discontinuity}}.  

Now, evaluate the following in order to make a conclusion about the type of discontinuity $f$ has at $x=-4$.  If a limit or function value does not exist, write DNE.

\begin{itemize}

\item $\displaystyle\lim_{x \to -4} f(x) = \answer{\frac{-1}{8}}$

\item $f(-4) = \answer{DNE}$

\end{itemize}

Based on this, you can conclude that at $x=-4$ $f$ has a(n) \wordChoice{\choice[correct]{removable discontinuity/hole}\choice{jump discontinuity}\choice{infinite discontinuity}\choice{oscillating discontinuity}}.  

\end{exercise}
\end{exercise}
\end{exercise}

\begin{exercise}

Consider the function $g(x) = 5x^4 - \pi x^2 - 5.$  Is this function continuous on $\RR = (\infty, \infty)?$  

\begin{multipleChoice}
    \choice[correct]{Yes}
    \choice{No}
\end{multipleChoice}

\begin{feedback}[correct]

Absolutely.  $g(x)$ is a polynomial, and polynomials are continuous on $(-\infty, \infty)$.

\end{feedback}

\end{exercise}

\begin{exercise}
Consider the piece-wise function 

\[
h(x) = \begin{cases}
  3x+5  & x<0 \\
  x^2 & x \geq 0
\end{cases}
\]

Is this function continuous on $\RR = (\infty, \infty)?$  

\begin{multipleChoice}
    \choice{Yes}
    \choice[correct]{No}
\end{multipleChoice}

\begin{exercise}

You got it: if you look closely, this function is discontinuous at $x=0$.  Select the reason why this function is discontinuous at $x=0$ below.  

\begin{multipleChoice}
    \choice{$h(0)$ does not exist}
    \choice[correct]{$\displaystyle\lim_{x \to 0} h(x)$ does not exist}
    \choice[correct]{$h(0) \neq \displaystyle\lim_{x \to 0} h(x)$}
    
\begin{feedback}[correct]
The main reason why $h(x)$ is discontinuous at $x=0$ is because $\displaystyle\lim_{x \to 0} h(x)$ does not exist.  You could also say that $h(0) \neq \displaystyle\lim_{x \to 0} h(x)$ because $h(0)$ exists and $\displaystyle\lim_{x \to 0} h(x)$ doesn't.
\end{feedback}

\end{multipleChoice}
\end{exercise}
\end{exercise}

\begin{exercise}
Consider the piece-wise function 

\[
j(x) = \begin{cases}
  x^3  & x < 1 \\
  5 & x=1 \\
  2x-1 & x > 1
\end{cases}
\]

Is this function continuous on $\RR = (\infty, \infty)?$  

\begin{multipleChoice}
    \choice{Yes}
    \choice[correct]{No}
\end{multipleChoice}

\begin{exercise}

Good thinking: this function is discontinuous at $x=1$.  Select the reason why this function is discontinuous at $x=1$ below.

\begin{multipleChoice}
    \choice{$j(1)$ does not exist}
    \choice{$\displaystyle\lim_{x \to 1} j(x)$ does not exist}
    \choice[correct]{$j(1) \neq \displaystyle\lim_{x \to 1} j(x)$}
\end{multipleChoice}

\begin{exercise}

Let's change $j(x)$ just slightly: 

\[
j(x) = \begin{cases}
  x^3  & x < 1 \\
  1 & x=1 \\ 
  2x-1 & x > 1
\end{cases}
\]

With this change, is this $j(x)$ continuous on $\RR = (\infty, \infty)?$  

\begin{multipleChoice}
    \choice[correct]{Yes}
    \choice{No}
    
\begin{feedback}[correct]
You'll notice that changing a single number resulted in $j(x)$ being continuous on $\RR$ because now $j(1) = \displaystyle\lim_{x \to 1} j(x)$.  The small details really do matter!   
\end{feedback}
\end{multipleChoice}

\end{exercise}
\end{exercise}
\end{exercise}

\begin{exercise}
Let 

\[
f(x) = \begin{cases}
  3x+1 & x < 1 \\
  \sqrt{A+2x}  & x \geq 1
\end{cases}
\]

where $A$ is a constant.  

When $A =\answer{14}$, $f(x)$ is continuous everywhere. 

\end{exercise}

\begin{exercise}
Let 

\[
z(x) = \begin{cases}
  Bx-2  & x < 4 \\
  Cx & x = 4 \\
   -\frac{1}{2} Bx +10 & x > 4
\end{cases}
\]

where $B$ and $C$ are constants. 

When $B =\answer{2}$ and $C = \answer{\frac{3}{2}}$, $z(x)$ is continuous everywhere.

\end{exercise}

\begin{exercise}

Consider the statement below, and then indicate whether it is sometimes, always, or never true.

\begin{center} ``If $f$ is continuous at $x=-2$, then $\displaystyle\lim_{x\to -2} f(x)$ exists." \end{center}

This statement is \wordChoice{\choice{sometimes}\choice[correct]{always}\choice{never}} true.

\begin{hint}

You may want to review the definition of continuity on \href{https://ximera.osu.edu/math160fa17/m160exam1content/limitLaws/digInContinuity}{this card} if you are struggling with this question. 

\end{hint}

\end{exercise}

\begin{exercise}

Consider the statement below, and then indicate whether it is sometimes, always, or never true.

\begin{center} ``If $\displaystyle\lim_{x\to -2} f(x)$ exists, then $f$ is continuous at $x=-2$." \end{center}

This statement is \wordChoice{\choice[correct]{sometimes}\choice{always}\choice{never}} true.

\begin{feedback}[correct]

Now that you know this statement is sometimes true, try to draw an example of a graph for which it is true and an example of a graph for which it is false.  

\end{feedback}

\end{exercise}

\begin{exercise}

Consider the statement below, and then indicate whether it is sometimes, always, or never true.

\begin{center} ``If $f$ is a continuous function on $(-5, 5)$, $f(-5) = -3$, and $f(5) = 3$, then there is an $x$-value $c$ in $(-5,5)$ such that $f(c) = 0$." \end{center}

This statement is \wordChoice{\choice[correct]{sometimes}\choice{always}\choice{never}} true.

\begin{feedback}[correct]

At first glance, this seems like an example of the Intermediate Value Theorem which would \textit{always} be true; however, there's a small (but important) difference between this statement and the IVT.  Can you spot it?

\end{feedback}

\end{exercise}


\end{document}
