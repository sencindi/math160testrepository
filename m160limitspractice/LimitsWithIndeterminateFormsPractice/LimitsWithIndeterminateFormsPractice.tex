\documentclass[handout]{ximera}
%\documentclass[10pt,handout,twocolumn,twoside,wordchoicegiven]{xercises}
%\documentclass[10pt,handout,twocolumn,twoside,wordchoicegiven]{xourse}

\input{../../preamble.tex}

\outcome{Practice Limits.}
   

\title{Limits with indeterminate forms practice}

\begin{document}
\begin{abstract}
Here is an opportunity for you to practice evaluating limits with indeterminate forms. 
\end{abstract}
\maketitle

\begin{exercise}
\[\lim_{z \to 0} \frac{x^2+x}{x^3-z} = \answer{-1}\]
\end{exercise}

\begin{exercise}
\[\lim_{q \to 0} \cfrac{q}{\frac{1}{q} - \frac{1}{q-1}} = \answer{0}\]
\end{exercise}

\begin{exercise}
\[\lim_{x \to 4} \frac{x-16}{\sqrt{x} + 4} = \answer{-2}\]
\end{exercise}

\begin{exercise}
\[\lim_{c \to 1} \cfrac{\frac{1}{c-1}}{\frac{1}{2c-2}} = \answer{2}\]
\end{exercise}

\begin{exercise}
\[\lim_{\alpha \to \pi} \sin(\alpha) \cos(\alpha) \cot(\alpha) = \answer{1}\]
\end{exercise}

\begin{exercise}
\[\lim_{a \to 0} \cfrac{\sqrt{x+4} -2}{x} = \answer{\frac{1}{4}}\]
\end{exercise}

\begin{exercise}
\[\lim_{x \to -5} \frac{x^2 + 4x -5}{x^2 + 7x + 10} = \answer{2}\]
\end{exercise}

\begin{exercise}
\[\displaystyle\lim_{\theta \to \frac{3\pi}{2}} \frac{\sin^2(\theta)\cos(\theta) + \cos^3(\theta)}{\cos(\theta)} = \answer{1}\]
\end{exercise}

\begin{exercise}
\[\lim_{l \to -3} \cfrac{1+\frac{3}{l}}{l+3} = \answer{\frac{-1}{3}}\]
\end{exercise}

\begin{exercise}

Consider the statement below, and then indicate whether it is sometimes, always, or never true.

\begin{center} ``If the limit $\displaystyle\lim_{x \to -5} f(x)$ is of the form $\frac{0}{0}$, then $\displaystyle\lim_{x\to -5} f(x)$ does not exist." \end{center}

This statement is \wordChoice{\choice[correct]{sometimes}\choice{always}\choice{never}} true.

\end{exercise}




\end{document}
