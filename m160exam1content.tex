\documentclass[10pt,handout,twocolumn,twoside,wordchoicegiven]{xourse}

\input{preamble.tex}

%% \iftikzexport
%% \newcommand\chapterstyle{}
%% \newcommand\sectionstyle{} 
%% \else
%% \usepackage{lulu1}
%% \fi
%\usepackage{lulu1}

\title{Math 160 Calculus 1 for Colorado State University: Exam 1 Content}

\begin{document}
\maketitle

\setcounter{tocdepth}{2}

%\part{Velocity at an Instant}

%% An application of limits
%\chapterstyle
%\activity{anApplicationOfLimits/titlePage.tex} 
%\sectionstyle
%\activity{anApplicationOfLimits/breakGround.tex}
%\activity{anApplicationOfLimits/digInInstantaneousVelocity.tex} (The function used on this xard is a big pain... should change)

%Note: We need some xards that deal with velocity at an instant that introduce the purpose of limits.  These old ones teach velocity as an application of limits. 

%\part{Tolerance}

\part{Introduction to Limits}

%% What is a limit
\chapterstyle
\activity{whatIsALimit/titlePage.tex}
\sectionstyle
\activity{whatIsALimit/breakGround.tex}
\activity{whatIsALimit/digInWhatIsALimit.tex}
\activity{m160limitspractice/WhatIsALimitExercises/WhatIsALimitExercises.tex}

%% Limit Laws
\chapterstyle
\activity{limitLaws/titlePage.tex}
\sectionstyle
\activity{limitLaws/breakGround.tex}

\activity{limitLaws/digInLimitLaws.tex}
\activity{limitLaws/digInTheSqueezeTheorem.tex}
\activity{m160limitspractice/LimitLawsExercises/LimitLawsExercises.tex}
  
%% Techniques for Evaluating Limits
\chapterstyle
\activity{indeterminateForms/titlePage.tex}
\sectionstyle
\activity{indeterminateForms/breakGround.tex}
\activity{indeterminateForms/digInLimitsOfTheFormZeroOverZero.tex}
\activity{m160limitspractice/LimitsWithIndeterminateFormsExercises/LimitsWithIndeterminateFormsExercises.tex}

\part{Limits of Piece-Wise Functions}

%% Limits of Piece-Wise Functions
\chapterstyle
\activity{limitsOfPieceWiseFunctions/titlePage.tex}
\sectionstyle
\activity{exercisesForMath160/ExPiecewiseLimits.tex}


\part{Continuity and the IVT}

%% Continuity and the intermediate value theorem
\chapterstyle
\activity{continuity/titlePage.tex}
\sectionstyle
\activity{continuity/breakGround.tex}
\activity{limitLaws/digInContinuity.tex}
\activity{continuity/digInContinuityOfPiecewiseFunctions.tex}
\activity{continuity/digInTheIntermediateValueTheorem.tex}
\activity{m160limitspractice/ContinuityExercises/ContinuityExercises.tex}

\part{Extrema and the EVT}

%% Extrema and the EVT
\chapterstyle 
\activity{extremaIntro/titlePage.tex}
\sectionstyle
\activity{extremaIntro/digInExtremaIntro.tex}

\part{Limits with Infinity}

%% Using limits to detect asymptotes
\chapterstyle
\activity{asymptotesAsLimits/titlePage.tex}
\sectionstyle
\activity{asymptotesAsLimits/breakGround.tex}
\activity{indeterminateForms/digInLimitsOfTheFormNonZeroOverZero.tex}
\activity{asymptotesAsLimits/digInVerticalAsymptotes.tex}
\activity{asymptotesAsLimits/digInHorizontalAsymptotes.tex}
\activity{m160limitspractice/LimitsWithInfinityExercises/LimitsWithInfinityExercises.tex}

\part{Mixed Limit Exercises}

%% Practice Problems
\sectionstyle
\activity{exercisesForMath160/ExMixedLimitsExercises1.tex}
\activity{exercisesForMath160/ExMixedLimitsExercises2.tex}
\activity{exercisesForMath160/ExMixedLimitsExercises3.tex}
\activity{exercisesForMath160/ExMixedLimitsExercises4.tex}
%\activity{exercisesForMath160/ExMixedLimitsPracticeBank.tex} (Note: This document houses a bank of problems to pick the mixed practice from.)

%\underline{\hspace{5in}}

%\iftikzexport\else\printindex\fi
\end{document}
