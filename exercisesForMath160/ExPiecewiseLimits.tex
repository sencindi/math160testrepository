\documentclass[handout]{ximera}
%\documentclass[10pt,handout,twocolumn,twoside,wordchoicegiven]{xercises}
%\documentclass[10pt,handout,twocolumn,twoside,wordchoicegiven]{xourse}

%\author{Steven Gubkin}
%\license{Creative Commons 3.0 By-NC}
\input{../preamble.tex}

\outcome{Practice limits with piecewise functions.}
   

\title[Exercises:]{Limits of Piecewise Functions Exercises}

\begin{document}
\begin{abstract}
  Here we'll practice finding limits with piecewise functions.
\end{abstract}
\maketitle

%%Problem 1
\begin{exercise}
Let
\[
f(x) =
\begin{cases} x^2+55 &\text{if $x<3$,}\\
  0 &\text{if $x=3$,} \\
  b^x &\text{if $x>3$.}
\end{cases}
\]  
What must the be the value of $b$ to make $\displaystyle\lim_{x \to 3} f(x)$ exist?

\[
b = \answer{4}
\]

\begin{hint}
  The left and righthand limits at $x=3$ must be equal.  Use this to
  set up an equation in terms of $b$, and solve for $b$.
\end{hint}
\end{exercise}

%%Problem 2
\begin{exercise}
Let
\[
g(x) = \begin{cases}
  \frac{x^3 - 8}{x-2}  &\text{if $x<1$,} \\
  x^3+1 &\text{if  $x>1$.}
\end{cases}
\]
Does $\displaystyle\lim_{x \to 2} g(x)$ exist?  If it does, give its value.
Otherwise write DNE.

\[
\lim_{x \to 2} g(x) = \answer{9}
\]

\begin{hint}
	Note that, close to $x=2$, the rule for $g(x)$ is $x^3+1$.
\end{hint}

\end{exercise}


%%Problem 3
\begin{exercise}
Let
\[
f(x) = \begin{cases}
  |x| &\text{if $x<1$,} \\
  \frac{x^2-a^2}{x-a} &\text{if $x>1$.}
\end{cases}
\]
If $\displaystyle\lim_{x \to 1} f(x)$ exists, what must be the value of $a$?

  \[
a = \answer{0}
\]

\begin{hint}
  Equate the two one sided limits of $f$ at $x=1$ to obtain an equation involving $a$.
\end{hint}

\end{exercise}

%%Problem 4
\begin{exercise}
Let $S(x) = \frac{|x|}{x}$.  Does $\displaystyle\lim_{x \to -4} S(x)$ exist?  If it
does, give its value.  Otherwise write DNE.

\[
\lim_{x \to -4} S(x) = \answer{-1}
\] 

\begin{hint}
  Close to $x=-4$, $S$ has the rule $\frac{-x}{x}$.
\end{hint}


\end{exercise}





















\end{document}
