\documentclass[handout]{ximera}
%\documentclass[10pt,handout,twocolumn,twoside,wordchoicegiven]{xercises}
%\documentclass[10pt,handout,twocolumn,twoside,wordchoicegiven]{xourse}

%\author{Steven Gubkin}
%\license{Creative Commons 3.0 By-NC}
\input{../preamble.tex}

\outcome{Practice limits with piece-wise functions.}
   

\title[Exercises:]{Limits of piece-wise functions exercises}

\begin{document}
\begin{abstract}
Here is an opportunity for you to practice finding one- and two-sided limits of piece-wise functions.
\end{abstract}
\maketitle

Before getting started, you may want to brush-up on what is meant by a piece-wise function and the notation of piece-wise functions.  You can do that \href{https://ximera.osu.edu/math160fa17/m160prerequisites/PreRequisiteXards/U2CommonFunctions/2.4PieceWiseFunctions/titlePage}{here}.

%%Problem 1
\begin{exercise}
Let
\[
f(x) = \begin{cases}
  \frac{x^3 - 8}{x-2}  &\text{if $x<1$,} \\
  x^3+1 &\text{if  $x>1$.}
\end{cases}
\]
Does $\displaystyle\lim_{x \to 2} f(x)$ exist?  If it does, give its value.
Otherwise write DNE.

\[
\lim_{x \to 2} f(x) = \answer{9}
\]

\begin{hint}
When $x$ is close to 2, what is the rule for $f(x)$?
\end{hint}

\end{exercise}

%%Problem 2 - Thought up by Brady
\begin{exercise}
Let
\[
g(x) = \begin{cases}
  \frac{x^2+5x}{x}  & x<0 \\
  1 & 0 \leq x \leq 1 \\
  \frac{x}{\sqrt{x-1} +1} & x>1
\end{cases}
\]
Use $g(x)$ to evaluate the following limits if they exist.  Otherwise, write DNE.

\begin{itemize}

\begin{multicols}{2}

\item [] $\lim_{x \to -2} g(x) = \answer{3}$

\item [] $\lim_{x \to 0^-} g(x) = \answer{5}$

\item [] $\lim_{x \to 0^+} g(x) = \answer{1}$

\item [] $\lim_{x \to 0} g(x) = \answer{DNE}$

\item [] $\lim_{x \to 1^-} g(x) = \answer{1}$

\item [] $\lim_{x \to 1^+} g(x) = \answer{1}$

\item [] $\lim_{x \to 1} g(x) = \answer{1}$

\item [] $\lim_{x \to 5} g(x) = \answer{\frac{5}{3}}$

\end{multicols}

\end{itemize}

\begin{hint}

To evaluate the 2-sided limits, you will first need to evaluate the corresponding 1-sided limits. 

\end{hint}

\end{exercise}

The next few problems will involve re-writing absolute value functions as piece-wise functions.  If you are unfamiliar with absolute value functions in general or need to review how to re-write absolute value functions as piece-wise functions, take a look at \href{https://www.youtube.com/watch?v=71SfBO-B4dE}{this link}.

%https://ximera.osu.edu/math160fa17/m160prerequisites/PreRequisiteXards/U2CommonFunctions/2.5TheAbsoluteValueFunction/titlePage

%%Problem 3
\begin{exercise}
Let $S(x) = \frac{|x|}{x}$.  Does $\displaystyle\lim_{x \to -4} S(x)$ exist?  If it
does, give its value.  Otherwise write DNE.

\[
\lim_{x \to -4} S(x) = \answer{-1}
\] 

\begin{hint}
When $x$ is close to $x=-4$, what is $|x|$ equal to? 
\end{hint}

\end{exercise}

%% Problem 4
\begin{exercise}
Let $f(t) = \frac{t^2 - 12t +35}{|t-7|}$.  Use $f(t)$ to evaluate the following limits if they exist.  Otherwise, write DNE.

\begin{itemize}

\item [] $\lim_{x \to 7^-} f(t) = \answer{-2}$

\item [] $\lim_{x \to 7^+} f(t) = \answer{2}$

\item [] $\lim_{x \to 7} f(t) = \answer{DNE}$

\end{itemize}

\begin{hint}
Absolute value functions are piece-wise functions, so you may want to re-write $f(t)$ as an explicit piece-wise function before trying to evaluate the limits. 
\end{hint}

\end{exercise}

%% Problem 5
\begin{exercise}
Let $K(x) = \frac{x^2}{|x|}$.  Use $K(x)$ to evaluate the following limit if it exists.  Otherwise, write DNE.

\[
\lim_{x \to 0} K(x) = \answer{0}
\]

\end{exercise}

%%Problem 6
\begin{exercise}
Let
\[
f(x) =
\begin{cases} x^2+55 &\text{if $x<3$,}\\
  0 &\text{if $x=3$,} \\
  b^x &\text{if $x>3$.}
\end{cases}
\]  
What must the be the value of $b$ to make $\displaystyle\lim_{x \to 3} f(x)$ exist?

\[
b = \answer{4}
\]

\begin{hint}
  The left- and right-hand limits at $x=3$ must be equal in order for $\displaystyle\lim_{x \to 3} f(x)$ to exist.  Use this information to
  set up an equation in terms of $b$, and then solve for $b$.
\end{hint}
\end{exercise}

%%Problem 7
\begin{exercise}
Let
\[
f(x) = \begin{cases}
  |x| &\text{if $x<1$,} \\
  \frac{x^2-a^2}{x-a} &\text{if $x>1$.}
\end{cases}
\]
If $\displaystyle\lim_{x \to 1} f(x)$ exists, what is $a$?

  \[
a = \answer{0}
\]

\end{exercise}





















\end{document}
