\documentclass[handout]{ximera}
%\documentclass[10pt,handout,twocolumn,twoside,wordchoicegiven]{xercises}
%\documentclass[10pt,handout,twocolumn,twoside,wordchoicegiven]{xourse}

%\author{Steven Gubkin}
%\license{Creative Commons 3.0 By-NC}
\input{../preamble.tex}

\outcome{Exercises: Mixed Limit Practice 2}
   

\title{Mixed limit practice 2}

\begin{document}
\begin{abstract}
Here is an opportunity for you to practice evaluating many different types of limits. 
\end{abstract}
\maketitle

%%limit law 1 - Mixed Practice 2%%
\begin{exercise}
Suppose that $\displaystyle\lim_{u\to2}c(u)=-5$, $\displaystyle\lim_{u\to2}B(u)=-1$, and $\displaystyle\lim_{u\to2}Y(u)=-4$. Compute the limit

\[
\lim_{u\to 2 } \frac{c(u)}{B(u)-Y(u)}\begin{prompt} = \answer{-\frac{5}{3}}\end{prompt}
\]
\end{exercise}

%%Trig 1 - MADE UP BY MARY - Mixed Practice 2%%
\begin{exercise}
\[\lim_{x \to 0} \frac{\tan(x)}{\sin(x)} = \answer{1}\]
\end{exercise}

%% Conjugate 3 - Mixed Practice 2
\begin{exercise}
\[
\lim_{\psi\to 4 } \frac{\sqrt{\psi +5}-3}{\psi -4}\begin{prompt} = \answer{\frac{1}{6}}\end{prompt}
\]
\end{exercise}

%% sinx/x 3 - Question 4 from exercises - Mixed Practice 2
\begin{exercise}
\[\lim_{x \to 0} \frac{\cos(x)-1}{x} = \answer{0}\]

\begin{hint}
This is a limit you could memorize.  Alternatively, if you've forgotten the value of this limit, you could multiply by $\cos(x)+1$ in the numerator and denominator and then apply the Pythagorean identity to get to the solution.
\end{hint}

\end{exercise}

%%Piece-wise limits 2 - Mixed Practice 2
\begin{exercise}
\[
k(x) = \begin{cases}
  \frac{\sqrt{x+3}-2}{x-1}  & x<1 \\
  7x & x=1 \\
  \frac{2x^3 -x}{4x} & x > 1
\end{cases}
\]
Does $\lim_{x \to 1} k(x)$ exist?  If it does, give its value.
Otherwise, write DNE.

\[
\lim_{x \to 1} k(x) = \answer{\frac{1}{4}}
\]
\end{exercise}

%%Inf Lim 3 - Mixed Practice 2
\begin{exercise}
Let 
\[
y(z) = \frac{5 z^2+z^2+z+5}{2 z^3-5 z^2-3 z+3}.
\]
Compute
\begin{enumerate}
\item $\displaystyle\lim_{z\to \infty} y(z) \begin{prompt} = \answer{0}\end{prompt}$
\item $\displaystyle\lim_{z\to -\infty}y(z) \begin{prompt} = \answer{0}\end{prompt}$
\end{enumerate}

\end{exercise}

%% DNE 1 - Mixed Practice 2
\begin{exercise}
Find
\[
\lim_{\theta\to\infty}\left(\cos(\theta)\right)
= \answer{DNE}
\]
\end{exercise}

%%Fractions 2 - MADE UP BY MARY - Mixed Practice 2%%
\begin{exercise}
\[\lim_{q \to 0} \frac{\frac{1}{7}-\frac{1}{3q}}{\frac{1}{3q}-\frac{1}{7}} = \answer{-1}\]
\end{exercise}











\end{document}
