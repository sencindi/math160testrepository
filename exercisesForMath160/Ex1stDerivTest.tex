\documentclass[handout]{ximera}
%\documentclass[10pt,handout,twocolumn,twoside,wordchoicegiven]{xercises}
%\documentclass[10pt,handout,twocolumn,twoside,wordchoicegiven]{xourse}

%\author{Steven Gubkin}
%\license{Creative Commons 3.0 By-NC}
\input{../preamble.tex}

\outcome{Exercises: First Derivative Test}
   

\title{First Derivative Test Exercises}

\begin{document}
\begin{abstract}
  Here we'll practice using the first derivative test.
\end{abstract}
\maketitle

%%PROBLEM 1
\begin{exercise}
The function $f(x) = x^3-6x+1$ has two critical points.  If we call these critical point $a$ and $b$, and order them such that $a < b$, then

$$
a = \answer{-\sqrt{2}}
$$

$$
b=\answer{\sqrt{2}}
$$

On $(\infty,a)$, $f$ is \wordChoice{\choice[correct]{increasing} \choice{decreasing}}

On $(a,b)$, $f$ is \wordChoice{\choice{increasing} \choice[correct]{decreasing}}

On $(b,\infty)$, $f$ is \wordChoice{\choice[correct]{increasing} \choice{decreasing}}


Consider $f$ restricted to the domain $[-3,2.5]$.

On this interval, the absolute maximum value of $f$ is $\answer{4\sqrt{2}+1}$.

On this interval, the absolute minimum value of $f$ is $\answer{-8}$.

\end{exercise}

%%PROBLEM 2
\begin{exercise}
The function $f(x) =\displaystyle\frac{1}{4\sqrt{x}}+x$ has only one critical point.  Call this critical point $a$ .

$$
a = \answer{0.25}
$$

On $(-\infty,a)$, $f$ is \wordChoice{\choice{increasing} \choice[correct]{decreasing}}

On $(a,\infty)$, $f$ is \wordChoice{\choice[correct]{increasing} \choice{decreasing}}

$(a,f(a))$ is an absolute \wordChoice{\choice{maximum} \choice[correct]{minimum}}

\end{exercise}

%%PROBLEM 3
\begin{exercise}
The function $f(x) =\frac{x^2-2x+1}{x(1+x^2)}$ has one critical
point and one point where $f(x)$ is undefined.  If we call these  points $a$ and $b$, and order them
such that $a < b$, then

$$
a = \answer{0}
$$

$$
b=\answer{1}
$$

On $(-\infty,a)$, $f$ is \wordChoice{\choice{increasing} \choice[correct]{decreasing}}

On $(a,b)$, $f$ is \wordChoice{\choice{increasing} \choice[correct]{decreasing}}

On $(b,\infty)$, $f$ is \wordChoice{\choice[correct]{increasing} \choice{decreasing}}


$x=a$ is a \wordChoice{\choice{Local max} \choice{Local Min} \choice[correct]{Neither}}

$x=b$ is a \wordChoice{\choice{Local max} \choice[correct]{Local Min} \choice{Neither}}

\end{exercise}


%%PROBLEM 4
\begin{exercise}
The function $f(x) =\frac{x^2+2x-3}{x+1}$ has two critical points and one point where the function is undefined.
If we call these points $a$, $b$, and $c$ and order them such
that $a < b < c$, then

$$
a = \answer{-3}
$$

$$
b=\answer{-1}
$$

$$
c=\answer{1}
$$



On $(-\infty,a)$, $f$ is \wordChoice{\choice[correct]{increasing} \choice{decreasing}}

On $(a,b)$, $f$ is \wordChoice{\choice{increasing} \choice[correct]{decreasing}}

On $(b,c)$, $f$ is \wordChoice{\choice{increasing} \choice[correct]{decreasing}}

On $(c,\infty)$, $f$ is \wordChoice{\choice[correct]{increasing} \choice{decreasing}}


$x=a$ is a \wordChoice{\choice[correct]{Local max} \choice{Local Min} \choice{Neither}}

$x=b$ is a \wordChoice{\choice{Local max} \choice{Local Min} \choice[correct]{Neither}}

$x=c$ is a \wordChoice{\choice{Local max} \choice[correct]{Local Min} \choice{Neither}}

\end{exercise}

%%PROBLEM 5
\begin{exercise}
The function $f(x) = 3x^4-8x^3+6x^2+7$ has two critical points.  If we
call these critical points $a$ and $b$, and order them such that $a <
b $, then

$$
a = \answer{0}
$$

$$
b=\answer{1}
$$



On $(-\infty,a)$, $f$ is \wordChoice{\choice{increasing} \choice[correct]{decreasing}}

On $(a,b)$, $f$ is \wordChoice{\choice[correct]{increasing} \choice{decreasing}}

On $(b,\infty)$, $f$ is \wordChoice{\choice[correct]{increasing} \choice{decreasing}}


$x=a$ is a \wordChoice{\choice{Local max} \choice[correct]{Local Min} \choice{Neither}}

$x=b$ is a \wordChoice{\choice{Local max} \choice{Local Min} \choice[correct]{Neither}}


\end{exercise}





















\end{document}