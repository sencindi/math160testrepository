\documentclass[handout]{ximera}
%\documentclass[10pt,handout,twocolumn,twoside,wordchoicegiven]{xercises}
%\documentclass[10pt,handout,twocolumn,twoside,wordchoicegiven]{xourse}

%\author{Steven Gubkin}
%\license{Creative Commons 3.0 By-NC}
\input{../preamble.tex}

\outcome{Exercises: Mixed Limit Practice 1}
   

\title{Mixed limit exercises 1}

\begin{document}
\begin{abstract}
Here is an opportunity for you to practice evaluating many different types of limits. 
\end{abstract}
\maketitle

%%Factoring 1 - MADE UP BY MARY - Mixed Practice 1%%
\begin{exercise}
\[\lim_{x \to 5} \frac{x^2-25}{x-5} = \answer{10}\]
\end{exercise}

%%Inf Lim 2 - Mixed Practice 1
\begin{exercise}
Let 
\[
F(z) = \frac{-5 z^3-3}{z^3-5 z^2-2 z}.
\]
Compute
\begin{enumerate}
\item $\displaystyle\lim_{z\to \infty} F(z) \begin{prompt} = \answer{-5}\end{prompt}$
\item $\displaystyle\lim_{z\to -\infty}F(z) \begin{prompt} = \answer{-5}\end{prompt}$
\end{enumerate}
\begin{hint}
Multiply by
\[
\frac{\frac{1}{z^3}}{\frac{1}{z^3}}
\]
\end{hint}
\end{exercise}

%%Fractions 1 - MADE UP BY MARY - Mixed Practice 1%%
\begin{exercise}
\[\lim_{x \to 2} \frac{\frac{1}{2}-\frac{1}{x}}{2-x} = \answer{-1/4}\]
\end{exercise}

%% sinx/x 1 - Question 1/2 from exercises - Mixed Practice 1
\begin{exercise}
\[\lim_{x \to 0} \frac{\sin(2x)}{3x} = \answer{2/3}\]
\begin{hint}
Recall that $\displaystyle\lim_{x \to 0} \frac{\sin x}{x} = 1$.
\end{hint}
\end{exercise}

%%piece-wise limits 1 - Mixed Practice 1%%
\begin{exercise}
\[
g(x) = \begin{cases}
  \frac{x^2 - 4}{x-2}  &\text{if $x<1$,} \\
  -x+1 &\text{if  $x>1$.}
\end{cases}
\]
Does $\lim_{x \to 2} g(x)$ exist?  If it does, give its value.
Otherwise, write DNE.

\[
\lim_{x \to 2} g(x) = \answer{-1}
\]

Does $\lim_{x \to 1} g(x)$ exist?  If it does, give its value.
Otherwise, write DNE.

\[
\lim_{x \to 1} g(x) = \answer{DNE}
\]
\end{exercise}

%%VA 1 - Mixed Practice 1%%
\begin{exercise}
Consider 
\[
f(\psi) = \frac{-5}{\psi -4}.
\]
Compute
\begin{enumerate}
\item $\displaystyle\lim_{\psi\to 4^- } f(\psi) \begin{prompt} = \answer{\infty}\end{prompt}$
\item $\displaystyle\lim_{\psi\to 4^+ } f(\psi) \begin{prompt} = \answer{-\infty}\end{prompt}$
\item $\displaystyle\lim_{\psi\to 4 } f(\psi) \begin{prompt} = \answer{DNE}\end{prompt}$
\end{enumerate}
\end{exercise}

%% Absolute Value 1 - Mixed Practice 1
\begin{exercise}
Find
\[
\lim_{x\to6}\left(\frac{\left|6-x\right|}{6-x}\right)
= \answer{DNE}
\]

\begin{hint}
Absolute value = piece-wise in disguise!
\end{hint}
\end{exercise}










\end{document}
