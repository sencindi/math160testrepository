\documentclass{ximera}

\input{../preamble.tex}

\title[Dig-In]{Maximums and minimums}

\outcome{Define absolute maximum and absolute minimum.}
\outcome{Find the absolute maximum and minimum using a graph.}
\outcome{Define relative maximum and relative minimum.}
\outcome{Find relative maxima and minima using a graph.}
\outcome{Compare and contrast relative and absolute maxima and minima.}
\outcome{Given a graph without an absolute maximum or minimum, explain why the graph has no absolute maximum or minimum.}
\outcome{Given a graph without any extrema, explain why the graph has no extrema.}
  

\begin{document}

\begin{abstract}
On this card, we investigate what is meant by the maxima and minima of a function.  
\end{abstract}
\maketitle

To motivate why you should care about maxima and minima, let's start with a realistic example. 

\begin{example}
Let's say that the company you work for sells fancy speaker systems.  As the resident mathematician, you've determined that your company's profit (in dollars) is given by 
\[ P(x) = -0.02x^2+150x-200000 \]
where $x$ represents the number of speakers your company produces.  How many speakers should your company produce in order to maximize profits? 
\begin{explanation}
From just the formula for the profit function, this question is seemingly pretty difficult, so let's plot the profit function. 

\[ \graph{-0.02x^2+150x-200000} \]
\end{explanation}
\end{example}




\end{document}
