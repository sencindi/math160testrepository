\documentclass{ximera}

\input{../preamble.tex}

\outcome{Identify situations in which a continuous function is guaranteed to have both an absolute maximum and minimum.}

\title[Dig-In:]{The Extreme Value Theorem}

\begin{document}
\begin{abstract}
On this card, we will determine the conditions that guarantee a continuous function has an absolute maximum and minimum. 
\end{abstract}
\maketitle

On the previous card, you filled in conclusions after each problem ithat involved identifying absolute extrema.  Use these conclusions to indicate whether the following statements are true or false.  We will then use your findings to state a theorem that identifies when a function is \textit{guaranteed} to have BOTH an absolute maximum and minimum.  

\begin{exercise}
\begin{center} ``A discontinuous function is guaranteed to have both an absolute maximum and minimum." \end{center}

This statement is \wordChoice{\choice{true}\choice[correct]{false}}.

\begin{feedback}[correct]
This statement is false.  For example, look back at $f(x)$ on the previous xard.
\end{feedback}
\end{exercise}

\begin{exercise}
\begin{center} A function on an open interval is guaranteed to have both an absolute maximum and minimum. \end{center}

This statement is \wordChoice{\choice{true}[correct]\choice[correct]{false}}.

\begin{feedback}[correct]
Right!  This statement is also false. Could you come up with an example of a function on an open interval, $(a,b)$, that does not have both an absolute maximum and minimum?
\end{feedback}
\end{exercise}

\begin{exercise}
\begin{center} ``If $f(x)$ is a function on a closed interval $[a,b]$, then $f$ is guaranteed to have both an absolute maximum and minimum." \end{center}

This statement is \wordChoice{\choice{true}\choice[correct]{false}}.

\begin{feedback}[correct]
This is false.  Although we have seen examples, like $g(x)$ on the last card, that do satisfy this statement, this statement is not \textit{always} true.  Can you come up with a counter-example that indicates why this statement is not true?
\end{feedback}
\end{exercise}

\begin{exercise}
\begin{center} ``If $f$ is a continuous function on a closed interval $[a,b]$, $f$ is guaranteed to have both an absolute maximum and minimum." \end{center}

This statement is \wordChoice{\choice[correct]{true}\choice{false}}.

\begin{feedback}[correct]
This statement is true!  In fact, this statement identifies the conditions we were looking for to guarantee a function has both an absolute maximum and minimum.  Let's give this statement a special name. 
\end{feedback}
\end{exercise}

\begin{theorem}[Extreme Value Theorem]\label{theorem:evt}\index{Extreme Value Theorem}
If $f$ is a continuous function on the closed interval
$[a,b]$, then $f$ attains both an absolute maximum value $M$ and an absolute minimum value $m$ on $[a,b]$.  That is, there are numbers $x_1$ and $x_2$ in $[a,b]$ with $f(x_1)=M$ and $f(x_2) = m$, and $m \leq f(x) \leq M$ for every other $x$ in $[a,b]$.

Below, we see a geometric interpretation of this theorem.
\begin{image}
\begin{tikzpicture}
	\begin{axis}[
            domain=0:6, xmin=0, xmax=6, ymin=0, ymax=2.5,
            axis lines =left, xlabel=$x$, ylabel=$y$,
            every axis y label/.style={at=(current axis.above origin),anchor=south},
            every axis x label/.style={at=(current axis.right of origin),anchor=west},
            xtick={1,2,4,5}, ytick={.2,2.2},
            xticklabels={$a$,$x_1$,$x_2$,$b$}, yticklabels={$f(x_2) = m$,$f(x_1) = M$},
            axis on top,
          ]
          \addplot [draw=none, fill=fill1, domain=(1:5)] {2.5} \closedcycle;

          \addplot [textColor,dashed] plot coordinates {(0,2.2) (2,2.2)};
          \addplot [textColor,dashed] plot coordinates {(0,.2) (4,.2)};
          \addplot [textColor,dashed] plot coordinates {(2,0) (2,2.2)};
          \addplot [textColor,dashed] plot coordinates {(4,0) (4,.2)};

          \addplot [fill1,very thick] plot coordinates {(1,0) (1,2.5)};
          \addplot [fill1,very thick] plot coordinates {(5,0) (5,2.5)};

          \addplot [very thick,penColor, smooth,domain=(1.5:2.5)] {sin(deg(x*1.57-1.57)) + 1.2};%max
          \addplot [very thick,penColor, smooth,domain=(3.5:4.5)] {sin(deg(x*1.57-1.57)) + 1.2};%min
          \addplot [very thick,dashed,penColor!50!background, smooth,domain=(2.5:3.5)] {sin(deg(x*1.57 - 1.57)) + 1.2}; 
          \addplot [very thick,dashed,penColor!50!background, smooth,domain=(1:1.5)] {sin(deg(x*1.57 - 1.57)) + 1.2}; 
          \addplot [very thick,dashed,penColor!50!background, smooth,domain=(4.5:5)] {sin(deg(x*1.57 - 1.57)) + 1.2}; 
          
          \addplot [color=penColor,fill=penColor,only marks,mark=*] coordinates{(2,2.2)};  %% closed hole          
          \addplot [color=penColor,fill=penColor,only marks,mark=*] coordinates{(4,.2)};  %% closed hole          
        \end{axis}
\end{tikzpicture}
%% \caption{A geometric interpretation of the Extreme Value Theorem. A
%%   continuous function $f(x)$ attains both an global maximum and an
%%   global minimum on an interval $[a,b]$. Note, it may be the case
%%   that $a=c$, $b=d$, or that $d<c$.}
%% \label{figure:extreme-value}
%% \end{marginfigure}
\end{image}
\end{theorem}


\end{document}
