\documentclass{ximera}

\input{../preamble.tex}

\outcome{Approximate a slant asymptote from the graph of a function.}

\title[Break-Ground:]{Zoom Out}

\begin{document}
\begin{abstract}
Two young mathematicians discuss what curves look like when one
``zooms out.''
\end{abstract}
\maketitle

Check out this dialogue between two calculus students (based on a true
story):

\begin{dialogue}
\item[Devyn] Riley, think about this function:
  \[
  f(x) = \frac{1}{x}.
  \]
\item[Riley] Hmmm. If you plot it, the graph looks like this:
\begin{image}
\begin{tikzpicture}
	\begin{axis}[
            domain=-4:4,
            ymax=10,
            ymin=-10,
            samples=100,
            axis lines =middle, xlabel=$x$, ylabel=$y$,
            every axis y label/.style={at=(current axis.above origin),anchor=south},
            every axis x label/.style={at=(current axis.right of origin),anchor=west}
          ]
	  \addplot [very thick, penColor, smooth, domain=(-3:-0.01)] {(1)/(x)};
      \addplot [very thick, penColor, smooth, domain=(0.01:3)] {(1)/(x)};
      \addplot [textColor, dashed] plot coordinates {(0,-20) (0,20)};
 
          
        \end{axis}
\end{tikzpicture}
\end{image}
\item[Devyn] Right! What I've noticed is that if $x$ gets really big or really small, then
  our function looks like a line.
\item[Riley] Yeah!  And if the $x$ values get really close to $0$, our function also looks like a different line! 
\item[Devyn] Whoa!  You're blowin' my mind. 
\end{dialogue}

\begin{problem}
Devyn and Riley have noticed that the function $f(x) = \frac{1}{x}$ looks like a horizontal line $y \ = \answer{0}$ for $x$ values far away from $0$ and looks like the vertical line $x \ = \answer{0}$ for $x$ values near $0$.  
\end{problem}

Let's investigate why this is the case using some tables.  \\

You may need to input the symbol for infinity to answer one of the following questions.  To do so, type $\verb|infty|$ or $\verb|infinity|$ or $\verb|oo|$.  

\begin{problem}
We've noticed that $f(x) = \frac{1}{x}$ looks like a horizontal line for $x$ values far away from $0$.  Fill in the tables below to answer the following question. 
  \[
  \begin{array}{l|l}
    x      & f(x) = \frac{1}{x}     \\ \hline
    1    & \begin{prompt}\answer{1}\end{prompt}\\
    10   & \begin{prompt}\answer{0.1}\end{prompt}\\
    100  & \begin{prompt}\answer{0.01}\end{prompt}\\
    1000 & \begin{prompt}\answer{0.001}\end{prompt} \\
    10000 & \begin{prompt}\answer{0.0001}\end{prompt} \\
    100000 & \begin{prompt}\answer{0.00001}\end{prompt} \\
  \end{array}
  \qquad\text{and}\qquad
  \begin{array}{l|l}
     x      & f(x) = \frac{1}{x}     \\ \hline
    -1    & \begin{prompt}\answer{-1}\end{prompt}\\
    -10   & \begin{prompt}\answer{-0.1}\end{prompt}\\
    -100  & \begin{prompt}\answer{-0.01}\end{prompt}\\
    -1000 & \begin{prompt}\answer{-0.001}\end{prompt} \\
    -10000 & \begin{prompt}\answer{-0.0001}\end{prompt} \\
    -100000 & \begin{prompt}\answer{-0.00001}\end{prompt} \\
    \end{array}
  \]
  
\begin{itemize}
  
\item In the first table, as the $x$ values became more positive, $y = \frac{1}{x}$ \wordChoice{\choice{increased}\choice[correct]{decreased}} toward $y = \answer{0}$.  Based on this, it appears that $\displaystyle\lim_{x \to \infty} \frac{1}{x} = \answer{0}$.

\item In the second table, as the $x$ values became more negative, $y = \frac{1}{x}$ \wordChoice{\choice[correct]{increased}\choice{decreased}} toward $y = \answer{0}$.  Based on this, it appears that $\displaystyle\lim_{x \to -\infty} \frac{1}{x} = \answer{0}$.

\end{itemize}

\end{problem}

\begin{problem}
We've also noticed that $f(x) = \frac{1}{x}$ looks like a vertical line for $x$ values near $0$.  Fill in the tables below to answer the following question. 
  \[
  \begin{array}{l|l}
    x      & f(x) = \frac{1}{x}     \\ \hline
    10    & \begin{prompt}\answer{0.1}\end{prompt}\\
    1  & \begin{prompt}\answer{1}\end{prompt}\\
    \frac{1}{10}  & \begin{prompt}\answer{10}\end{prompt}\\
    \frac{1}{100} & \begin{prompt}\answer{100}\end{prompt} \\
    \frac{1}{1000} & \begin{prompt}\answer{1000}\end{prompt} \\
    \frac{1}{10000} & \begin{prompt}\answer{10000}\end{prompt} \\
  \end{array}
  \qquad\text{and}\qquad
  \begin{array}{l|l}
    x      & f(x) = \frac{1}{x}     \\ \hline
    -10    & \begin{prompt}\answer{-0.1}\end{prompt}\\
    -1  & \begin{prompt}\answer{-1}\end{prompt}\\
    -\frac{1}{10}  & \begin{prompt}\answer{-10}\end{prompt}\\
    -\frac{1}{100} & \begin{prompt}\answer{-100}\end{prompt} \\
    -\frac{1}{1000} & \begin{prompt}\answer{-1000}\end{prompt} \\
    -\frac{1}{10000} & \begin{prompt}\answer{-10000}\end{prompt} \\
  \end{array}
  \]
  
\begin{itemize}
  
\item In the first table, as the $x$ values got closer to $0$, $y = \frac{1}{x}$ rapidly \wordChoice{\choice[correct]{increased}\choice{decreased}}.  Based on this, it appears that $\displaystyle\lim_{x \to 0^+} \frac{1}{x} = \answer{\infty}$.

\item In the second table, as the $x$ values got closer to $0$, $y = \frac{1}{x}$ rapidly \wordChoice{\choice{increased}\choice[correct]{decreased}}.  Based on this, it appears that $\displaystyle\lim_{x \to 0^-} \frac{1}{x} = \answer{-\infty}$.

\end{itemize}

\end{problem}


%\input{../leveledQuestions.tex}


\end{document}
