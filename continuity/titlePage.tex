\documentclass{ximera}

\input{../preamble.tex}

\title{Continuity and the IVT}

\begin{document}

\begin{abstract}
%Stuff can go here later if we want!
\end{abstract}

\maketitle

\begin{sectionOutcomes}
After completing this section, you should be able to do the following.

\begin{itemize}
\item State the three components of the limit definition of continuity. 
\item Identify the locations of discontinuities of a function. 
\item Identify the intervals on which a function is continuous.
\item Identify which common functions are continuous on their domains.
\item Make a piece-wise function continuous.
\item State the Intermediate Value Theorem (hypotheses and conclusion).
\item Determine if the Intermediate Value Theorem applies to a given situation.
\item Sketch pictures indicating why the Intermediate Value Theorem is
  true and why all hypotheses are necessary.
\item Explain why certain points exist using the Intermediate Value
  Theorem.
\end{itemize}

\phantom{text}%%% Making vertical spaces


%---Prereq notes/links----------%
\textbf{Skills you may want to brush up on first} \\ To be ready to achieve these objectives, you may need to review the following trigonometry and algebra topics: 
\begin{itemize}
    \item Common types of functions: power functions, \href{https://ximera.osu.edu/math160fa17/m160prerequisites/PreRequisiteXards/U2CommonFunctions/2.1Polynomials/2.1TitlePage/2.1TitlePage}{polynomial functions}, and \href{https://ximera.osu.edu/math160fa17/m160prerequisites/PreRequisiteXards/U2CommonFunctions/2.2RationalFunctions/titlePage}{rational functions}
        \item 
  \href{https://ximera.osu.edu/math160fa17/m160prerequisites/PreRequisiteXards/U1Functions/1.3CompositionOfFunctions/titlePage}{Composition of Functions}
\end{itemize}

\end{sectionOutcomes}


\end{document}
