\documentclass{ximera}

\input{../../../../preamble.tex}

\title{Pre-Quiz}

\begin{document}
\begin{abstract}
Answer the following questions honestly to assess your own abilities. This pre-quiz will not be graded. It can be used for you to determine your own strengths and weaknesses.
\end{abstract}
\maketitle

\begin{problem} 
    Could you rewrite the sum of fractions $\frac{4}{3} + \frac{3}{4}$ as a single fraction (AKA add the fractions)? 
  \begin{multipleChoice}
      \choice[correct]{Yes, could add these two fractions together.}
      \choice[correct]{Yes, I could, but I would need to review some resources first.}
      \choice[correct]{No, I would need to learn/relearn how to do this.}
      \choice[correct]{I don't know.}
      \choice[correct]{I don't want to answer this question right now.}
  \end{multipleChoice}
\end{problem}

\begin{problem} 
    Could you list the following fractions in increasing order?
    $$\frac{\sqrt{3}}{2}, 0, 1, \frac{\sqrt{3}}{4}$$
  \begin{multipleChoice}
      \choice[correct]{Yes, can list the above fractions in increasing order.}
      \choice[correct]{Yes, I could, but I would need to review some resources first.}
      \choice[correct]{No, I would need to learn/relearn how to do this.}
      \choice[correct]{I don't know.}
      \choice[correct]{I don't want to answer this question right now.}
  \end{multipleChoice}
\end{problem}

\begin{problem} 
    What is $\frac{3}{4} + \frac{4}{3}$ equal to? (Mark all correct answers)
    
  \begin{selectAll}
      \choice{$\frac{7}{7}$}
      \choice{$1$}
      \choice[correct]{$\frac{25}{12}$}
      \choice{$\frac{12}{12}$}
      \choice{$\frac{7}{12}$}
      \choice{$1 + \frac{5}{7}$}
      \choice{$2$}
      \choice[correct]{$2 + \frac{1}{12}$}
      \choice{I don't know.}
      \choice{I don't want to answer this question right now.}
  \end{selectAll}
\end{problem}

\begin{problem} 
    List the following fractions in increasing order
    $$\frac{\sqrt{3}}{2}, 0, 1, \frac{\sqrt{3}}{4}$$
  \begin{multipleChoice}
      \choice{$0, 1, \frac{\sqrt{3}}{2}, \frac{\sqrt{3}}{4}$}
      \choice{$0, \frac{\sqrt{3}}{2}, 1, \frac{\sqrt{3}}{4}$}
      \choice{$0, \frac{\sqrt{3}}{4}, 1, \frac{\sqrt{3}}{2}$}
      \choice{$0, \frac{\sqrt{3}}{2}, \frac{\sqrt{3}}{4}, 1$}
      \choice[correct]{$0, \frac{\sqrt{3}}{4}, \frac{\sqrt{3}}{2}, 1$}
      \choice{The correct order is not given.}
      \choice{I don't know.}
      \choice{I don't want to answer this question right now.}
  \end{multipleChoice}
\end{problem}

\begin{problem} 
    Did you implement any strategies when completing the content quiz questions? \\ (Mark all that apply)
  \begin{selectAll}
      \choice[correct]{A. I answered the questions in order, assuming that the easier ones would \\ come first.}
      \choice[correct]{B. I answered the questions that I knew how to do first.}
      \choice[correct]{C. I read through all of the questions first before answering any.}
      \choice[correct]{D. After reading a question, I worked backwards and used process of elimination \\ to narrow down my options by checking if the answers were correct.}
      \choice[correct]{E. I drew pictures to get a geometric understanding of adding fractions.}
      \choice[correct]{F. I found a common denominator.}
      \choice[correct]{G. I used online resources to find the solutions.}
      \choice[correct]{H. I used online resources to remind me how to answer the questions, but then solved them myself.}
      \choice[correct]{I. I used a calculator to find the answers.}
      \choice[correct]{J. I used a calculator to check the solutions I found by hand.}
      \choice[correct]{K. I used strategies when completing the content quiz, but I don't remember the names of the strategies.}
      \choice[correct]{L. I used other strategies that aren’t listed here.}
      \choice[correct]{M. I didn't use any strategies when completing the content quiz.}
  \end{selectAll}
\end{problem}

\begin{problem}
    Please list any additional strategies that you used when taking this the content quiz. If you didn't use any additional strategies, just type `NA.`
   \begin{freeResponse}
   \end{freeResponse}
\end{problem}



%input{..leveledQuestions.tex}


\end{document}
