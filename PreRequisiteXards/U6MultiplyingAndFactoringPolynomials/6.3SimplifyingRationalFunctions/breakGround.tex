\documentclass{ximera}

\input{../../../preamble.tex}

\outcome{Distinguish two functions by considering their domains.}
\outcome{Recognize different representations of the same function.}

\title[Break-Ground:]{Same or different? Take 2}

\begin{document}
\begin{abstract}
  Two young mathematicians examine one (or two?)  functions once again.
\end{abstract}
\maketitle

Check out this dialogue between two calculus students (based on a true
story):

\begin{dialogue}
\item[Devyn] Riley, I have another pressing question.
\item[Riley] Sure, what is it?
\item[Devyn] Think about the function
\[ f(x) = \frac{x^2-3x+2}{x-2}. \]
\item[Riley] Okay.  What about it?
\item[Devyn] Is this function equal to $g(x) = x-1$?
\item[Riley] Well if I plot them with my calculator, they look the
  same.
\item[Devyn] I know!  But \href{https://ximera.osu.edu/math160fa17/m160prerequisites/understandingFunctions/breakGround}{remember when we realized before that $\frac{2x}{x}$ and $2$ are not the same function?}  I think this could be a similar situation.
\item[Riley] Okay, let's check.  I suppose I could write
  \begin{align*}
    f(x) &= \frac{x^2-3x+2}{x-2} \\
    &= \frac{(x-1)(x-2)}{x-2} \\
    &= x-1 \\
    &= g(x).
  \end{align*}
\item[Devyn] Right! But what about when $x=1$? In this case,
  \[
  g(1) = 0\qquad\text{but}\qquad f(0) \text{ is undefined!}
  \]
\item[Riley] Okay, $f(0)$ is undefined because we cannot divide by
  zero. Hmm... 
\end{dialogue}

\begin{problem}
  In the context above, are $f$ and $g$ the same function?
  \begin{multipleChoice}
    \choice{yes}
    \choice[correct]{no}
  \end{multipleChoice}
\end{problem}



\begin{problem}
  Suppose $f$ and $g$ are functions but the domain of $f$ is different
  from the domain of $g$.  Could it be that $f$ and $g$ are actually
  the same function?

  \begin{multipleChoice}
    \choice{yes}
    \choice[correct]{no}
  \end{multipleChoice}

  \begin{feedback}
    The domain of a function is part of the ``data'' of the function.
    A function is not a rule for transforming the input to the output,
    but rather the relationship between a specified collection of
    inputs (the domain) and possible outputs (the range).
  \end{feedback}
\end{problem}

\begin{problem}
  If you simplify a function $f$ and realize the simplified form is $g$, is it possible that $f$ and $g$ have different domains and are therefore different functions?

  \begin{multipleChoice}
    \choice[correct]{yes}
    \choice{no}
  \end{multipleChoice}

  \begin{feedback}[correct]
  When simplifying functions, you have to be careful because sometimes the simplified form of a function does not have the same domain as the original function.  In this case, the two functions are not actually equal!  We will investigate this further in the next card.
  \end{feedback}
\end{problem}


 
%\input{../leveledQuestions.tex}


\end{document}
