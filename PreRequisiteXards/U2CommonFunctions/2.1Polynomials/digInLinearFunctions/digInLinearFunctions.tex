\documentclass{ximera}

\input{../../../../preamble.tex}


\title[Dig-In:]{[NOT READY YET] Linear functions}


\begin{document}
\begin{abstract}
On this card, we will review linear functions and a couple of ways of writing linear functions. 
\end{abstract}
\maketitle

\textbf{The following card is still under construction but it should be available by the end of the first week of classes.  Hit the update button above around then, and you should see a complete card.  In the meantime, \href{http://tutorial.math.lamar.edu/Classes/Alg/Lines.aspx}{here} is a useful resource for reviewing linear functions.}

\section{What determines a line?}

There are infinitely many lines, so how do we distinguish between them in order to talk about a single line?  Let's investigate. 

\begin{exercise}
Imagine that you are discussing some math homework with a friend over the phone.  You are discussing a line, and you are trying to describe which line you are talking about to your friend.  What information could you give to your friend to make sure that he/she knows which line you are talking about?  (Select all the possibilities below)

\begin{selectAll}
    \choice{1 point on the line}
    \choice[correct]{2 points on the line}
    \choice[correct]{3 points on the line}
    \choice{The slope of the line}
    \choice[correct]{One point on the line and the slope of the line}
    \choice[correct]{The $y$-intercept of the line and the slope of the line}
\end{selectAll}

\begin{explanation}
The minimal amount of information needed to describe a line is with 2 points OR with 1 point and the slope of the line.  When writing the equation of a line or a linear function, though, we most often use 1 point on the line and the slope of the line.  
\end{explanation}

\section{What is the slope of a line?}

One piece of information that can be used to specify a line or linear function is the slope of a line.  But what is the slope of a line?  Let's take a look at some examples.  

\begin{example}
Consider the linear function $y = 3x$ graphed below.  
\begin{image}
\begin{tikzpicture}
	\begin{axis}[
            domain=-5:5,
            ymax=20,
            ymin=-20,
            samples=100,
            axis lines =middle, xlabel=$x$, ylabel=$y$,
            every axis y label/.style={at=(current axis.above origin),anchor=south},
            every axis x label/.style={at=(current axis.right of origin),anchor=west}
          ]
	  \addplot [very thick, penColor, smooth, domain=(-5:5)] {3*x};
         
       \end{axis}
\end{tikzpicture}
\end{image}

When $x=0$, then $y=0$ as well.  Therefore, the $y$-intercept is $0$.  If $x$ increases by 1, what will happen to the $y$ value?  

\begin{multipleChoice}
    \choice{It will increase by 1.}
    \choice{It will decrease by 1.}
    \choice[correct]{It will increase by 3.}
    \choice{It will decrease by 3.}
    \choice{It will increase by $\frac{1}{3}$.}
    \choice{It will decrease by $\frac{1}{3}$.}
\end{multipleChoice}

\begin{explanation}
You got it.  Looking at the graph $y=3x$, you can see that if $x$ increases by 1, then $y$ will increase by $3$.  Therefore, when $x=1$, the corresponding $y$ value will be $0+3 = 3$, so the point $(1,3)$ is on the line $y=3x$. 
\end{explanation}

%\begin{question}
%Now we know $(1,3)$ is on this line.  Let's say the $x$ value is increased by $2$ so that it becomes $3$.  What happens to the $y$ value in this situation?  

%\begin{multipleChoice}
    %\choice{It will increase by 2*1.}
    %\choice{It will decrease by 2*1.}
    %\choice[correct]{It will increase by 2*3.}
    %\choice{It will decrease by 2*3.}
    %\choice{It will increase by $2*\frac{1}{3}$.}
    %\choice{It will decrease by $2*\frac{1}{3}$.}
%\end{multipleChoice}

%\end{question}
\end{example}

\end{exercise}



\end{document}
