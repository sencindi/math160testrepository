\documentclass{ximera}

\input{../../../../preamble.tex}

\title{Trig Functions}

\begin{document}
\begin{abstract} 
%Stuff can go here later if we want!
\end{abstract}

\maketitle

\begin{sectionOutcomes}
After completing this section, students should be able to do the following.

\begin{itemize}
	\item Use the Pythagorean Theorem to calculate the side length of a right triangle when given the other two side lengths. 
    \item State the definition of $\sin(\theta), \cos(\theta), \tan(\theta)$ in the context of a right triangle.
    \item Calculate $\sin(\theta), \cos(\theta),$ and $\tan(\theta)$ using a right triangle with angle $\theta$.
    \item State the definition of a radian.
    \item Convert between degrees and radians.
    \item Use similar triangles to explain why it does not matter which specific right triangle is used to calculate the value of a trigonometric function.
    \item Explain how the $(x,y)$ points of the unit circle correspond to values of $(\cos(\theta), \sin(\theta))$ by using right triangles with a radius of one.
    \item State the definition of $\sin(\theta), \cos(\theta), \tan(\theta), \csc(\theta), \sec(\theta), \cot(\theta)$ in the context of the unit circle.
    \item Evaluate the six trigonometric functions at common angles using the unit circle.
    \item State the three Pythagorean Identities and their domains. 
    \item Derive the first Pythagorean Identity using a right triangle, the unit circle, and the Pythagorean Theorem.
    \item Identify the domains of inverse trigonometric functions.
    \item Use inverse trigonometric functions to solve simple trigonometric equations. 

\end{itemize}
\end{sectionOutcomes}

\end{document}
