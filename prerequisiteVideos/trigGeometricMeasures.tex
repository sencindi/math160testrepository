\documentclass{ximera}

\input{../preamble.tex}
\title[Prerequisite Videos: ]{Trigonometry: Areas and Lengths}
\author{Ben Sencindiver}

\outcome{Understand area and length of circles and related shapes.}

\begin{document}
\begin{abstract}
  In this series of videos, we will talk about geometric measures and will
  serve as a guide to areas and lengths related to circles.
\end{abstract}
\maketitle

The following videos talk about area and length of shapes related to circles:

%% Introduction Video
\textbf{Introduction to geometric measures: What is the circumference
of a circle?}
\begin{explanation}
%% Labeling this expandable option
\begin{flushright}
{\color{blue}(\emph{Click the arrow to the right to see the Introduction video.})}
\end{flushright}
\begin{center}
\begin{expandable}
\youtube{SNEil7RRgbE}
\end{expandable}
\end{center}
\end{explanation}

%% Question 1
\textbf{Question 1: Circumference of circles}
\begin{question}
%% Labeling this expandable option
\begin{flushright}
{\color{blue}(\emph{Click the arrow to the right to see the first question.})}
\end{flushright}
\begin{center}
\begin{expandable}
\youtube{NrnHkPJ8OTU}
%% Multiple Choice Question 1
{\color{blue}(\emph{Click the arrow to the right to see the question
posed at the end of the video.})}
\begin{expandable}
Using that $\pi\approx 3.14$, what is the circumference of a circle with radius $3$ feet?\\
\begin{prompt}
The circumference of a circle is about $\answer[tolerance=0.001]{18.84}$ feet.
\end{prompt}
%% Example 1
\begin{flushright}
{\color{blue}(\emph{Click the arrow to the right to see an example.})}
\end{flushright}
\begin{expandable}
Example 1
\youtube{o7evtSaQTLE}
\end{expandable}
\end{expandable}
\end{expandable}
\end{center}
\end{question}


%% Question 2
\textbf{Question 2: Not quite the circumference of circles}
\begin{question}
%% Labeling this expandable option
\begin{flushright}
{\color{blue}(\emph{Click the arrow to the right to see the first question.})}
\end{flushright}
\begin{center}
\begin{expandable}
\youtube{m7j5VfypoFo}
%% Multiple Choice Question 2
{\color{blue}(\emph{Click the arrow to the right to see the question
posed at the end of the video.})}
\begin{expandable}
What is \emph{half} of the circumference of a circle with radius $r$?
\begin{prompt}
Half of the circumference of a circle with radius $r$ is exactly $\answer[given]{\pi r}$.
\end{prompt}
%% Example 2
\begin{flushright}
{\color{blue}(\emph{Click the arrow to the right to see an example.})}
\end{flushright}
\begin{expandable}
Example 2
\youtube{-PvmN4gJ6UM}
\end{expandable}
\end{expandable}
\end{expandable}
\end{center}
\end{question}


%% Arc Length Introduction
\textbf{Arc Length: The length of the arc subtended by an angle}
\begin{explanation}
%% Labeling this expandable option
\begin{flushright}
{\color{blue}(\emph{Click the arrow to the right to see the Introduction video.})}
\end{flushright}
\begin{center}
\begin{expandable}
\youtube{ltVJjniw0dE}
\end{expandable}
\end{center}
\end{explanation}


%% Question 3
\textbf{Question 3: Arc Length}
\begin{question}
%% Labeling this expandable option
\begin{flushright}
{\color{blue}(\emph{Click the arrow to the right to see the third question.})}
\end{flushright}
\begin{center}
\begin{expandable}
\youtube{7n2o8osdpKY}
%% Multiple Choice Question 3
{\color{blue}(\emph{Click the arrow to the right to see the question
posed at the end of the video.})}
\begin{expandable}
What is the arc length of an arc subtended by the angle $225^\circ$
for a circle with radius $3$ meters?\\
\begin{prompt}
The arc length is exactly $\answer[given]{\frac{15\pi}{4}}$.
\end{prompt}
%% Example 3
\begin{flushright}
{\color{blue}(\emph{Click the arrow to the right to see an example.})}
\end{flushright}
\begin{expandable}
Example 3
\youtube{OaJttvXjG4k}
\end{expandable}
\end{expandable}
\end{expandable}
\end{center}
\end{question}


\end{document}
