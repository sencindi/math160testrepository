\documentclass{ximera}

\input{../preamble.tex}
\title[Prerequisite Videos: ]{[Video] Piece-Wise Functions}
\author{Ben Sencindiver}

\outcome{Compute function values of piece-wise functions, find domain/range of piece-wise functions, and graph piece-wise functions.}

\begin{document}
\begin{abstract}
   In this set of videos, we will work to understand how to find function values of piece-wise functions, find the domain and range of piece-wise functions, and how to graph piece-wise functions.
\end{abstract}
\maketitle

The following videos will cover topics about piece-wise functions.

%% Question 1
\textbf{Question 1: What is the multiplicity of a polynomial at a root?}
\begin{question}
%% Labeling this expandable option
\begin{flushright}
{\color{blue}(\emph{Click the arrow to the right to see the first question.})}
\end{flushright}
\begin{center}
\begin{expandable}
\youtube{0DhaITXMAdw}
%% Multiple Choice Question 1
\begin{flushright}
{\color{blue}(\emph{Click the arrow to the right to see the question
posed at the end of the video.})}
\end{flushright}
\begin{expandable}
The function $f(x) = 17x^4(x+1)^3(x-1)^2(x-2)$ has multiplicity what at root $x=1$?
\begin{multipleChoice}
\choice{$0$}
\choice{$1$}
\choice[correct]{$2$}
\choice{$3$}
\choice{$4$}
\end{multipleChoice}
%% Example 1
\begin{flushright}
{\color{blue}(\emph{Click the arrow to the right to see an example.})}
\end{flushright}
\begin{expandable}
Example 1
\youtube{ts9lN_FhXAM}
\end{expandable}
\end{expandable}
\end{expandable}
\end{center}
\end{question}


%% Question 2
\textbf{Question 2: Matching polynomial functions to their graphs.}
\begin{question}
%% Labeling this expandable option
\begin{flushright}
{\color{blue}(\emph{Click the arrow to the right to see the second question.})}
\end{flushright}
\begin{center}
\begin{expandable}
\youtube{4yfHu4pIxck}
%% Multiple Choice Question 2
\begin{flushright}
{\color{blue}(\emph{Click the arrow to the right to see the  question
posed at the end of the video.})}
\end{flushright}
\begin{expandable}
Which of the following graphs is the graph of the function $f(x) = \frac{1}{6}x^2(x-2)^2(x-1)(x+1)^2(x+2)$?

\begin{multicols}{3}
\begin{center}
A. \includegraphics[scale=0.25]{Graph1}
\end{center}

\begin{center}
B. \includegraphics[scale=0.25]{Graph2}
\end{center}
 
\begin{center}
C. \includegraphics[scale=0.25]{Graph3}
\end{center}
\end{multicols}

\begin{multicols}{3}
\begin{center}
D. \includegraphics[scale=0.25]{Graph4}
\end{center}

\begin{center}
E. \includegraphics[scale=0.25]{Graph5} %Correct One
\end{center}

\begin{center}
F. \includegraphics[scale=0.25]{Graph6}
\end{center}
\end{multicols}

\begin{multipleChoice}
\choice{$A$}
\choice{$B$}
\choice{$C$}
\choice{$D$}
\choice[correct]{$E$}
\choice{$F$}
\end{multipleChoice}
%% Example 2
\begin{flushright}
{\color{blue}(\emph{Click the arrow to the right to see an example.})}
\end{flushright}
\begin{expandable}
Example 2
\youtube{a1TxdPwfhXY}
\end{expandable}
\end{expandable}
\end{expandable}
\end{center}
\end{question}


%% Question 3
\textbf{Question 3: Matching polynomial graphs to their functions.}
\begin{question}
%% Labeling this expandable option
\begin{flushright}
{\color{blue}(\emph{Click the arrow to the right to see the third question.}) }
\end{flushright}
\begin{center}
\begin{expandable}
\youtube{cXHoAiLJ990}
%% Multiple Choice Question 3
\begin{flushright}
{\color{blue}(\emph{Click the arrow to the right to see the question
posed at the end of the video.})}
\end{flushright}
\begin{expandable}
Which of the following functions is the graph below?
\begin{multipleChoice}
\choice{$A$}
\choice{$B$}
\choice{$C$}
\choice{$D$}
\choice{$E$}
\end{multipleChoice}
%% Example 3
\begin{flushright}
{\color{blue}(\emph{Click the arrow to the right to see an example.})}
\end{flushright}
\begin{expandable}
Example 3
\youtube{Lzkn37a21HU}
\end{expandable}
\end{expandable}
\end{expandable}
\end{center}
\end{question}


%% Question 4
\textbf{Question 4: Factoring polynomials using their graphs.}
\begin{question}
%% Labeling this expandable option
\begin{flushright}
{\color{blue}(\emph{You may want to review \href{https://ximera.osu.edu/math160fa17/m160prerequisites/PreRequisiteXards/U6MultiplyingAndFactoringPolynomials/6.2FactoringPolynomials/titlePage}{factoring polynomials} before working on question 4.  Once you are prepared to work on question 4, click the arrow to the right.})}
\end{flushright}
\begin{center}
\begin{expandable}
\youtube{KQnvwEs4H8E}
%% Multiple Choice Question 4
\begin{flushright}
{\color{blue}(\emph{Click the arrow to the right to see the question
posed at the end of the video.})}
\end{flushright}
\begin{expandable}
Given the graph of the function $f(x) = \frac{1}{5}x^5 + \frac{1}{5}x^4 - 2x^3 - \frac{4}{5}x^2 + \frac{24}{5}x$ below, factor $f(x)$
\begin{center}
\includegraphics[scale=0.4]{GraphFactor1.png}
\end{center}
\begin{multipleChoice}
\choice{$\frac{4}{5}x(x-2)(x+2)(x+3)$}
\choice{$\frac{1}{5}x(x-2)(x+2)^2(x+3)$}
\choice[correct]{$\frac{1}{5}x(x-2)^2(x+2)(x+3)$}
\choice{$\frac{1}{5}x(x-2)(x+2)(x+3)$}
\choice{$\frac{4}{5}x(x-2)^2(x+2)(x+3)$}
\choice{$\frac{1}{5}x^3(x-2)^4(x+2)^3(x+3)$}
\end{multipleChoice}
%% Example 4
\begin{flushright}
{\color{blue}(\emph{Click the arrow to the right to see an example.})}
\end{flushright}
\begin{expandable}
Example 4
\youtube{2kX7nwT3zY0}
\end{expandable}
\end{expandable}
\end{expandable}
\end{center}
\end{question}

\end{document}

