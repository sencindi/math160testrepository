\documentclass{ximera}

\input{../preamble.tex}
\title[Prerequisite Videos: ]{[Video] Piece-Wise Functions}
\author{Ben Sencindiver}

\outcome{Find function values of piece-wise functions, find domain/range of piece-wise functions, and graph piece-wise functions.}

\begin{document}
\begin{abstract}
   In this set of videos, we will work to understand how to find function values of piece-wise functions, how to find the domain and range of piece-wise functions, and how to graph piece-wise functions.
\end{abstract}
\maketitle

%% Question 1
\textbf{Question 1: Finding function values of piece-wise functions.}
\begin{question}
%% Labeling this expandable option
\begin{flushright}
{\color{blue}(\emph{Click the arrow to the right to see the first question.})}
\end{flushright}
\begin{center}
\begin{expandable}
\youtube{Egm75HO7gzs}
%% Multiple Choice Question 1
\begin{flushright}
{\color{blue}(\emph{Click the arrow to the right to see the question
posed at the end of the video.})}
\end{flushright}
\begin{expandable}
Find $f(1)$ when $f(x)$ is defined by
\[ f(x) = \begin{cases} 
      \frac{1}{2}x + \frac{3}{2} & x<1 \\
     x^2+3x - 6 & 1\leq x\leq 10 \\
      (x-1)(x+1) & x>10\\	
   \end{cases}. \]
\begin{multipleChoice}
\choice[correct]{$-2$}
\choice{$-1$}
\choice{$0$}
\choice{$1$}
\choice{$2$}
\choice{$f(1)$ is undefined}
\end{multipleChoice}
%% Example 1
\begin{flushright}
{\color{blue}(\emph{Click the arrow to the right to see an example.})}
\end{flushright}
\begin{expandable}
\begin{center}
Example 1
\end{center}
\youtube{bkArouEU3g0}
\end{expandable}
\end{expandable}
\end{expandable}
\end{center}
\end{question}


%% Question 2
\textbf{Question 2: Finding the domain of a piece-wise function.}
\begin{question}
%% Labeling this expandable option
\begin{flushright}
{\color{blue}(\emph{Click the arrow to the right to see the second question.})}
\end{flushright}
\begin{center}
\begin{expandable}
\youtube{J6I0Jcaqiaw}
%% Multiple Choice Question 2
\begin{flushright}
{\color{blue}(\emph{Click the arrow to the right to see the  question
posed at the end of the video.})}
\end{flushright}
\begin{expandable}
What is the domain of $f(x)$ if $f(x)$ is defined by the expression below?\\

\[ f(x) = \begin{cases} 
      2x+13 & x<-3 \\
      \dfrac{x^2+1}{x+5} & -3< x\leq 1 \\
      \frac{-1}{x^2-4} & x> 1\\	
   \end{cases} \]

\begin{multipleChoice}
\choice{$(-\infty, -3)$}
\choice{$(-\infty, -3]$}
\choice{$(-3, 1)$}
\choice{$(-3, 1]$}
\choice{$[-3,-1]$}
\choice{$(1, \infty)$}
\choice{$[1, \infty)$}
\choice{$(-\infty, -3) \cup (-3, 1]$}
\choice{$(-3, 1] \cup (1, \infty)$}
\choice{$(-\infty, -3) \cup (-3, 1] \cup [1, \infty)$}
\choice{$(-\infty, -3) \cup (-3, -2) \cup (-2, 1] \cup (1, 2) \cup (2, \infty)$}
\choice[correct]{$(-\infty, -3) \cup (-3, 1] \cup (1, 2) \cup (2, \infty)$}
\end{multipleChoice}

Hint: Are there any values of $x>1$ for which $\frac{-1}{x^2-4}$ is undefined?

%% Example 2
\begin{flushright}
{\color{blue}(\emph{Click the arrow to the right to see an example.})}
\end{flushright}
\begin{expandable}
\begin{center}
Example 2
\end{center}
\youtube{BeeaOn6zSO8}
\end{expandable}
\end{expandable}
\end{expandable}
\end{center}
\end{question}


%% Question 3
\textbf{Question 3: Find the range of a piece-wise function.}
\begin{question}
%% Labeling this expandable option
\begin{flushright}
{\color{blue}(\emph{Click the arrow to the right to see the third question.}) }
\end{flushright}
\begin{center}
\begin{expandable}
\youtube{QLNZW-_cLeM}
%% Multiple Choice Question 3
\begin{flushright}
{\color{blue}(\emph{Click the arrow to the right to see the question
posed at the end of the video.})}
\end{flushright}
\begin{expandable}

What is the range of $f(x)$ when $f(x)$ is defined by the expression below?\\

\[ f(x) = \begin{cases} 
      2x+3 & x<-3 \\
      x^2 & -3\leq x< 2 \\
      -1 & x> 3\\	
   \end{cases} \]
   
\begin{multipleChoice}
\choice{$(\infty, -3) \cup [0,9]$}
\choice{$(\infty, -3] \cup [4,9]$}
\choice{$(\infty, -1] \cup [0,9]$}
\choice[correct]{$(\infty, -3) \cup \{-1\} \cup [0,9]$}
\choice{$(\infty, -3) \cup \{-1\} \cup [0,9)$}
\end{multipleChoice}
%% Example 3
\begin{flushright}
{\color{blue}(\emph{Click the arrow to the right to see an example.})}
\end{flushright}
\begin{expandable}
\begin{center}
Example 3
\end{center}
\youtube{tjn-DyPJhQA}
\end{expandable}
\end{expandable}
\end{expandable}
\end{center}
\end{question}


%% Question 4
\textbf{Question 4: Graphing piece-wise functions.}
\begin{question}
%% Labeling this expandable option
\begin{flushright}
{\color{blue}(\emph{Click the arrow to the right to see the fourth question.})}
\end{flushright}
\begin{center}
\begin{expandable}
\youtube{_3zhjq1uyEc}
%% Multiple Choice Question 4
\begin{flushright}
{\color{blue}(\emph{Click the arrow to the right to see the question
posed at the end of the video.})}
\end{flushright}
\begin{expandable}
Which of the graphs is the graph of $f(x)$ where $f(x)$ is defined below?
\[ f(x) = \begin{cases} 
      x^2-1 & x \leq 0 \\
     \frac{1}{3} x + 2 & 0< x \leq 3 \\
    4 & x>3.5
   \end{cases} \]

\begin{multipleChoice}
\choice{A.\includegraphics[scale=0.2]{PWGraph1}}
\choice{B.\includegraphics[scale=0.2]{PWGraph6}}
\choice[correct]{C.\includegraphics[scale=0.2]{PWGraph2}}
\choice{D.\includegraphics[scale=0.2]{PWGraph5}}
\choice{E.\includegraphics[scale=0.2]{PWGraph4}}
\choice{None of the above.}
\end{multipleChoice}
%% Example 4
\begin{flushright}
{\color{blue}(\emph{Click the arrow to the right to see an example.})}
\end{flushright}
\begin{expandable}
\begin{center}
Example 4
\end{center}
\youtube{dq24Kyw7XTU}
\end{expandable}
\end{expandable}
\end{expandable}
\end{center}
\end{question}

\end{document}

