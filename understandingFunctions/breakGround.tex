\documentclass{ximera}

\input{../preamble.tex}

\outcome{Distinguish two functions by considering their domains.}
\outcome{Recognize different representions of the same function.}

\title[Break-Ground:]{Same or different?}

\begin{document}
\begin{abstract}
  Two young mathematicians examine one (or two?)  functions.
\end{abstract}
\maketitle

Check out this dialogue between two calculus students (based on a true
story):

\begin{dialogue}
\item[Devyn] Riley, I have a pressing question.
\item[Riley] Tell me. Tell me everything.
\item[Devyn] Think about the function
  \[
  f(x) = \frac{2x}{x}.
  \]
\item[Riley] OK.
\item[Devyn] Is this function equal to $g(x) = 2$?
\item[Riley] Well if I plot them with my calculator, they look the
  same.
\item[Devyn] I know!
\item[Riley] And I suppose if I write
  \begin{align*}
    f(x) &= \frac{2x}{x} \\
    &= \frac{2\cancel{x}}{\cancel{x}} \\
    &= 2 \cdot 1 \\
    &= 2 \\
    &= g(x).
  \end{align*}
\item[Devyn] Sure! But what about when $x=0$? In this case
  \[
  g(0) = 2\qquad\text{but}\qquad f(0) \text{ is undefined!}
  \]
\item[Riley] Right, $f(0)$ is undefined because we cannot divide by
  zero. Hmm. Now I see the problem. Yikes!
\end{dialogue}

\begin{problem}
  In the context above, are $f$ and $g$ the same function?
  \begin{multipleChoice}
    \choice{yes}
    \choice[correct]{no}
  \end{multipleChoice}
\end{problem}



\begin{problem}
  Suppose $f$ and $g$ are functions but the domain of $f$ is different
  from the domain of $g$.  Could it be that $f$ and $g$ are actually
  the same function?

  \begin{multipleChoice}
    \choice{yes}
    \choice[correct]{no}
  \end{multipleChoice}

  \begin{feedback}
    The domain of a function is part of the ``data'' of the function.
    A function is not a rule for transforming the input to the output,
    but rather the relationship between a specified collection of
    inputs (the domain) and possible outputs (the range).
  \end{feedback}
\end{problem}


\begin{problem}
  Can the same function be represented by different formulas?

  \begin{multipleChoice}
    \choice[correct]{yes}
    \choice{no}
  \end{multipleChoice}

  \begin{problem}
    Are $f(x) = 2x+1$ and $g(x) = 2(x-1)+3$ the same function?

    \begin{multipleChoice}
      \choice[correct]{These are the same function although they are represented by different formulas.}
      \choice{These are different functions because they have different formulas.}
    \end{multipleChoice}
  \end{problem}
\end{problem} 

\begin{problem}
	Let $f(x) =x^2$ and $g(u) = u^2$.  The domain of each of these functions is all real numbers.  Which of the following statements are true?
	\begin{multipleChoice}
	  \choice{There is not enough information to determine if $f = g$.}
	  \choice[correct]{The functions are equal.}
	  \choice{If $x \neq u$, then $f \neq g$.}
	  \choice{We have $f \neq g$ since $f$ uses the variable $x$ and $g$ uses the variable $u$.}
	\end{multipleChoice}
    
    \begin{feedback}[correct]
    Although one function is stated in terms of $x$ and the other is stated in terms of $u$, they are both fundamentally the same function: $f(1) = g(1)$, $f(-4) = g(-4)$, and, in general $f(c) = g(c)$ for \textit{every} value of $c$ in the real numbers.  You could also graph both of these functions, and they would have the same shape.  The mathematical relation $f(x)$ conveys is the same one that $g(u)$ conveys, so we say that $f = g$.  
    \end{feedback}
\end{problem}



%\input{../leveledQuestions.tex}


\end{document}
